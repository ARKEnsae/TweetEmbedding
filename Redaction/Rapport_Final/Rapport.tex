\documentclass[11pt,french,french]{article}
\usepackage{lmodern}
\usepackage{amssymb,amsmath}
\usepackage{ifxetex,ifluatex}
\usepackage{fixltx2e} % provides \textsubscript
\ifnum 0\ifxetex 1\fi\ifluatex 1\fi=0 % if pdftex
  \usepackage[T1]{fontenc}
  \usepackage[utf8]{inputenc}
\else % if luatex or xelatex
  \ifxetex
    \usepackage{mathspec}
    \usepackage{xltxtra,xunicode}
  \else
    \usepackage{fontspec}
  \fi
  \defaultfontfeatures{Mapping=tex-text,Scale=MatchLowercase}
  \newcommand{\euro}{€}
\fi
% use upquote if available, for straight quotes in verbatim environments
\IfFileExists{upquote.sty}{\usepackage{upquote}}{}
% use microtype if available
\IfFileExists{microtype.sty}{%
\usepackage{microtype}
\UseMicrotypeSet[protrusion]{basicmath} % disable protrusion for tt fonts
}{}
\usepackage[margin=0.90in]{geometry}
\ifxetex
  \usepackage{polyglossia}
  \setmainlanguage{}
\else
  \usepackage[shorthands=off,french]{babel}
\fi
\usepackage{longtable,booktabs}
\ifxetex
  \usepackage[setpagesize=false, % page size defined by xetex
              unicode=false, % unicode breaks when used with xetex
              xetex]{hyperref}
\else
  \usepackage[unicode=true]{hyperref}
\fi
\hypersetup{breaklinks=true,
            bookmarks=true,
            pdfauthor={},
            pdftitle={},
            colorlinks=true,
            citecolor=blue,
            urlcolor=blue,
            linkcolor=magenta,
            pdfborder={0 0 0}}
\urlstyle{same}  % don't use monospace font for urls
\setlength{\parindent}{0pt}
\setlength{\parskip}{6pt plus 2pt minus 1pt}
\setlength{\emergencystretch}{3em}  % prevent overfull lines
\setcounter{secnumdepth}{5}

\providecommand{\tightlist}{%
  %\setlength{\itemsep}{0pt}
  \setlength{\parskip}{0pt}
  }

%%% Use protect on footnotes to avoid problems with footnotes in titles
\let\rmarkdownfootnote\footnote%
\def\footnote{\protect\rmarkdownfootnote}


  \title{~Word-Embedding et sentiments des ménages avec Twitter}
    \author{Kim Antunez, Romain Lesauvage, Alain Quartier-la-Tente\\
sous l'encadrement de Benjamin Muller (Inria)}
    \date{}
  
\usepackage{caption}
\usepackage{graphicx}
\usepackage{natbib}
\usepackage[dvipsnames]{xcolor}
\usepackage{fontawesome5}
\DeclareMathOperator{\arctanh}{arctanh}
\usepackage{subcaption}
\usepackage{amsfonts}
\usepackage{dsfont}
\usepackage{xspace}
\usepackage{enumitem}
\usepackage{pifont}
\usepackage{wrapfig}
\usepackage{textpos}


\usepackage[tikz]{bclogo}
\newcounter{comptEncadre}
\renewcommand\thecomptEncadre{%\thesection.
\arabic{comptEncadre}}
\definecolor{processblue}{cmyk}{0.96,0,0,0}
\newenvironment{encadre}[2][false]{\refstepcounter{comptEncadre}
      %\addcontentsline{exp}{encadres}{\protect\numberline{\thecomptEncadre}#1}%
\begin{bclogo}[couleur=processblue!5,arrondi=0.1,
logo=\bcloupe,barre=none,couleurBord=blue!60!green,nobreak = #1]{ {\sc \textbf{Encadré \thecomptEncadre}} -  #2}
\smallskip
}{\end{bclogo}}

\begin{document}

\maketitle


{
\hypersetup{linkcolor=black}
\setcounter{tocdepth}{2}
\tableofcontents
}
\begin{textblock*}{\textwidth}(0cm,-24.3cm)
\begin{center}
\includegraphics[height=2.5cm]{img/LOGO-ENSAE.png}
\end{center}
\end{textblock*}

\newpage

\begin{center}
\begin{minipage}{0.7\textwidth}

\begin{wrapfigure}{L}{0cm}
\includegraphics[height=2.5cm]{img/LOGO-ENSAE.png}
\end{wrapfigure}

$\phantom{saut}$

$\phantom{saut}$

$\phantom{saut}$


\emph{Retrouvez l'ensemble du projet et son code sur}

https://github.com/ARKEnsae/TweetEmbedding

\end{minipage}
\end{center}

\vspace{0.5cm}

\hypertarget{introduction}{%
\section*{Introduction}\label{introduction}}
\addcontentsline{toc}{section}{Introduction}

Grâce à l'évolution des méthodes d'apprentissage profond (\emph{Deep Learning}), l'appréhension du langage naturel est aujourd'hui devenue une discipline à part entière (\emph{Natural Language Processing}).
Ce succès s'explique en partie grâce à l'émergence de techniques non supervisées d'apprentissage de représentation de structures linguistiques.
Les méthodes de \emph{word embedding} («~plongement lexical~» en français) permettent de représenter chaque mot d'un dictionnaire par un vecteur de nombres réels afin que les mots qui apparaissent dans des contextes similaires possèdent des vecteurs correspondants qui sont relativement proches (au sens d'une distance définie).
Les modèles \emph{word2vec}, développés par une équipe de recherche chez Google (\cite{Mikolov}), sont parmi les plus célèbres et sont ceux sur lesquels se concentrera notre projet.

Dans ce projet de statistique appliquée, nous étudierons dans un premier temps en détail et implémenterons le modèle \emph{word2vec} (partie \ref{sec:word2vec}).
Dans un deuxième temps, nous évaluerons la validité du modèle implémenté et l'appliquerons sur une base de données composée de plusieurs millions de tweets publiés en France entre 2013 et 2017 (partie \ref{sec:evaluation}).
Enfin, nous mobiliserons des techniques d'analyse de sentiment afin de créer des indicateurs qui pourront être comparés aux indicateurs produits dans la statistique publique, en particulier concernant l'opinion des ménages (partie \ref{sec:sentimentalAnalysis}).

\bigskip

\bigskip

\begin{figure}[htb]
\tikzstyle{myboxnorm} = [very thick,
    rectangle, rounded corners, inner sep=2pt, inner ysep=3pt, right,
    align=center]
\tikzstyle{myboxw2v} = [draw=red!50, fill=orange!10, myboxnorm]
\tikzstyle{myboxmod} = [draw=blue!60!green, fill=blue!5, myboxnorm]
\tikzstyle{myboxcomp} = [draw=green!50!black, fill=green!5, myboxnorm]

\tikzstyle{titlenorm} = [fill=white, very thick,
    rectangle, inner sep=2pt, inner ysep=2pt,font=\bfseries,
    text width=4.2cm, above=-0.3cm, align=center,
    minimum height=1.15cm]
\tikzstyle{titlew2v} = [draw=red!50, text = red!50,titlenorm]
\tikzstyle{titlemod} = [draw=blue!60!green,text=blue!60!green,titlenorm]
\tikzstyle{titlecomp} = [draw=green!50!black,text=green!50!black, titlenorm]    

\tikzstyle{fleche} = [->,rounded corners,line width=1pt]

\begin{tikzpicture}
\usetikzlibrary{fit}
\usetikzlibrary{arrows.meta}
%Il faut d'abord faire les boites
% w2vec
\node[fit={(2.5,1.5) (-3.,-3.5)}, myboxw2v,semitransparent] (w2vrect) {};
\node[fit={(2.7,1.5) (8.2,-5.5)}, myboxmod,semitransparent] (rectmod) {};
\node[fit={(13.9,1.5) (8.4,-5.5) }, myboxcomp,semitransparent] (rectcomp) {};

\node at (0,0) [myboxw2v] (tweets) {Tweets};
\node at (2.4,0) [myboxw2v] (we) {\textit{word-embedding}};
% sentiment
\node at (0,-4.5) [myboxnorm,text width=2.5cm, draw = black] (bddsent) {Échantillon de tweets annotés
};

% analyse de sentimentsentiment
\node at (7,0) [myboxmod] (logit) {Modèle logit};
\node at (6.8,-4.5) [myboxmod,text width=2.5cm] (baseline) {Modèle lexical};

% analyse de sentimentsentiment
\node at (12,0) [myboxcomp,text width=3.8cm] (indsent) {Indices mensuels de \\sentiment des tweets};

\node at (12,-4.5) [myboxcomp,text width=3.8cm] (camme) {Indicateur synthétique \\de confiance des

ménages (Insee)};

%Fleches
\draw[fleche] (tweets.east) --(we.west) node[below = 0.2cm,pos=1]{
\begin{minipage}{5cm} \footnotesize
  \begin{itemize}[label=\scalebox{.6}{\ding{110}}] 
  \item tokénisation
  \item choix hyperparamètres
  \item évaluation : 
  \begin{itemize}[label=\scalebox{.6}{\ding{117}}]
    \item similarité cosinus
    \item réduction de dimension (ACP/TSNE)
    \item jugement humain
  \end{itemize}
  \end{itemize} 
\end{minipage} 
};
\draw[fleche] (bddsent.east)--(baseline.west) 
  node[pos=0.36,text width=3cm]{
    \footnotesize bases  
    
    test/entraînement}
  node[pos=0.99, below = 0.05cm,text width=2.5cm]{
    \footnotesize sentiment 
    
    moyen d'un mot};
\draw[fleche] (bddsent.east)--++(2,0)--++(0,0.8)-|(logit.south);
\draw[fleche] (we.east)--(logit.west)node[pos=0.89,text width=2.5cm]{\footnotesize \emph{sentence}
  
  \emph{embedding}};
\draw[fleche] (logit.east)--
  (indsent.west)node[pos=0.57,text width=2.5cm]{\footnotesize moyennes
  
  mensuelles};
\draw[fleche] (baseline.east)-|(9.57,0)--(indsent.west);
\draw[fleche,<->] (indsent.south)--(camme.north) 
  node[pos=.5,text width=2.5cm, right=0.2cm]{\footnotesize distance entre
  
  indicateurs}
  node[pos=.5,text width=1.7cm, left]{\footnotesize prévision /
  
  causalité};

\node[titlew2v] at (w2vrect.north) { word2vec \\
(skip-gram)};
\node[titlemod] at (rectmod.north) { Analyse de \\
sentiment d'un tweet};
\node[titlecomp] at (rectcomp.north) { Comparaison \\
d'indices mensuels};
\end{tikzpicture}
\end{figure}

\newpage

\hypertarget{sec:word2vec}{%
\section{\texorpdfstring{Implémentation du modèle \emph{word2vec}}{Implémentation du modèle word2vec}}\label{sec:word2vec}}

\hypertarget{le-moduxe8le-word2vec-un-moduxe8le-de-word-embedding}{%
\subsection{\texorpdfstring{Le modèle \emph{word2vec}, un modèle de \emph{word-embedding}}{Le modèle word2vec, un modèle de word-embedding}}\label{le-moduxe8le-word2vec-un-moduxe8le-de-word-embedding}}

Le \emph{Natural Language Processing} (NLP ou «~traitement automatique du langage naturel~») est une branche du \emph{machine learning} visant à analyser, traiter et reproduire le langage humain.
Les modèles de NLP \emph{word2vec}, développés par une équipe de recherche chez Google (\cite{Mikolov}), sont parmi les plus célèbres et utilisent le \emph{word-embedding} -- plongement lexical en français.

\hypertarget{historique-de-la-suxe9mantique-vectorielle-uxe0-word2vec}{%
\subsubsection{\texorpdfstring{Historique : de la sémantique vectorielle à \emph{word2vec}}{Historique : de la sémantique vectorielle à word2vec}}\label{historique-de-la-suxe9mantique-vectorielle-uxe0-word2vec}}

La «~sémantique vectorielle~» est née dans les années 1950\footnote{L'ouvrage \cite{Jurafsky} permet de retracer avec une grande richesse l'évolution des méthodes de NLP.}.
C'est une méthode algébrique de représentation d'un document visant à réaliser des tâches diverses (détecter le plagiat, filtrer des articles\dots).
Il est alors nécessaire de capter de nombreux types de proximités entre mots : les synonymes (automobile / voiture), antonymes (froid / chaud), connotations positives \emph{versus} négatives (heureux / triste), etc.

Un modèle répondant à toutes ces exigences ne peut exister.
Pour y répondre au mieux, la sémantique vectorielle puise son inspiration des travaux linguistiques des années 1950 et en particulier de l' «~hypothèse de distribution~» selon laquelle un mot se définit par son environnement.
Dit autrement : les mots qui se produisent dans un contexte identique tendent à avoir des significations similaires\footnote{Comme l'a écrit le linguiste britannique John Rupert Firth en 1957, «~Vous connaîtrez un mot par ses fréquentations~».}.

Les premiers modèles sémantiques (comme le \emph{term frequency-inverse document frequency} (TF-IDF)) représentaient les relations entre mots grâce à des très grandes matrices, dites \emph{sparses}, dont les dimensions correspondaient à la taille du vocabulaire (contenant donc beaucoup de 0). Les méthodes de \emph{word-embedding} qui sont ensuite apparues ont permis de représenter chaque mot d'un dictionnaire par un vecteur de nombres réels denses (peu de 0) de plus faible dimension (en général entre 50 et 1000).
Si la réduction de dimension rend les vecteurs-mots moins facilement interprétables, elle a pour grand avantage de faciliter et d'accélérer les tâches d'apprentissage impliquant ces mots.

\cite{Mikolov} ont mis en avant en 2013 les méthodes de \emph{word-embedding} à travers la création de \emph{word2vec}. Ce modèle de réseaux de neurones\footnote{C'est l'article \cite{Bengio} qui a introduit dix ans avant \emph{word2vec} le premier modèle d'apprentissage de représentation de vecteurs-mots à partir d'un réseau de neurone simple.} à deux couches est rapidement devenu une référence grâce à la grande précision des résultats qu'il permet d'obtenir, pouvant être entraîné en un temps record sur un corpus très volumineux.

\hypertarget{subsec:word2vec}{%
\subsubsection{\texorpdfstring{\emph{word2vec}, un modèle d'apprentissage «~auto-supervisé~»}{word2vec, un modèle d'apprentissage «~auto-supervisé~»}}\label{subsec:word2vec}}

En sortie du modèle \emph{word2vec}, chaque mot est représenté par un vecteur dont la dimension est fixée par la valeur d'un hyperparamètre. Les mots qui apparaissent dans des contextes similaires («~bonjour~» et «~salut~» par exemple) seront représentés par des vecteurs relativement proches dans l'espace vectoriel de définition de ces vecteurs. Dans la même logique, \emph{word2vec} permet également de réaliser des opérations vectorielles, comme dans l'exemple, souvent cité : \(\overrightarrow{Paris} - \overrightarrow{France} + \overrightarrow{Italie} = \overrightarrow{Rome}\) qui provient de \cite{Mikolov}.

Deux architectures du modèle \emph{word2vec} existent (graphique \ref{fig:cbowskipgram}) :

\begin{itemize}
\item
  L'approche \emph{Continuous bags of words} dont l'objectif est d'estimer la probabilité d'observer un mot, appelé \textbf{«~focus~»}, sachant le contexte dans lequel il apparaît (i.e. : les mots \textbf{voisins} qualifiés de \textbf{«~contextes~»}).
\item
  L'approche \emph{Skip-gram} a un objectif inverse : estimer, pour chaque mot du vocabulaire, la probabilité d'être proche du mot focus.
  C'est cette approche que nous étudions dans ce projet et dans la suite de ce rapport.
\end{itemize}

\begin{figure}[htp]
\centering
\includegraphics[width=.6\textwidth]{img/cbow_skip_gram.png}
\captionsetup{margin=0cm,format=hang,justification=justified}
\caption{Architecture des modèles Continuous bags of words (CBOW) et Skip-gram.}\label{fig:cbowskipgram}
%\vspace{-0.3cm}
\footnotesize
\emph{Source : \cite{Mikolov}}
\end{figure}

Pour transformer chaque mot en un vecteur, au lieu de simplement compter les fréquences d'apparition des mots contextes voisins d'un mot focus\footnote{Comme dans les premiers modèles sémantiques dits \emph{sparses}.}, nous entraînons un réseau de neurones sur une tâche annexe : on construit un classifieur dont la tâche de prédiction est binaire pour chacun des mots du vocabulaire et répond à la question (dans le cas Skip-Gram) «~Est-ce que ce mot contexte est susceptible d'être voisin du mot focus ?~».
La prédiction en elle-même ne nous intéresse pas, c'est plutôt le poids du classifieur en sortie du modèle qui correspondra aux \emph{word-embeddings}.

Les voisins d'un mot focus reposent sur un hyperparamètre : la fenêtre (\emph{window} ou \(w\)).
Pour \(w = p\), les voisins du mot focus sont les \(p\) mots précédents et les \(p\) mots suivants dans la phrase. Par exemple, dans la phrase :

\begin{quote}
\LARGE \textbf{``}\normalsize \emph{Le professeur de statistique est strict avec ses élèves.} \LARGE \textbf{''}\normalsize
\end{quote}

pour \(w=2\), si le mot focus est «~statistique~» alors le contexte qui lui est associé est : \texttt{{[}professeur,\ de,\ est,\ strict{]}} ; si le mot focus est «~professeur~» alors le contexte qui lui est associé est : \texttt{{[}Le,\ de,\ statistique{]}}.

Pour déterminer les représentations vectorielles des mots, nous entraînons le réseau de neurones en le nourrissant des paires \texttt{{[}focus,\ contexte{]}}\footnote{Dans notre exemple : \texttt{{[}statistique,\ professeur{]},\ {[}statistique,\ de{]}}\dots} contenues dans les différentes phrases (ici tweets) du corpus afin qu'il puisse déterminer les probabilités d'apparition d'un mot dans le voisinage d'un autre mot (voir description de l'algorithme en partie \ref{sec:skipgram}).

Ainsi, la grande force du modèle d'apprentissage \emph{word2vec} est qu'il est «~auto-supervisé~». En effet, comme nous avons vu plus haut, le corpus est considéré comme une donnée d'entraînement implicitement supervisée, ce qui nous évite d'avoir à mobiliser des corpus annexes annotés.

\hypertarget{sec:skipgram}{%
\subsection{L'algorithme Skip-Gram}\label{sec:skipgram}}

L'objectif de cette partie est de décrire le fonctionnement de l'approche Skip-gram.

Dans la suite de ce projet nous noterons \(n\) la taille du vocabulaire (i.e. : le nombre de mots différents) et \(dim\) la dimension retenue pour les \emph{word-embeddings}.
Comme décrit dans la partie \ref{subsec:word2vec}, l'approche \emph{Skip-gram} peut être vue comme un réseau de neurones à deux couches avec :

\begin{itemize}
\item
  En entrée une matrice \(W_e\) de taille \(n\times dim\) ;
\item
  En sortie une matrice \(W_S\) de taille \(n\times dim\).
\end{itemize}

Ces deux matrices sont initialisées en générant des lois normale \(\mathcal N(0,1)\). Elles sont ensuite mises à jour, grâce aux couples \texttt{{[}focus,\ contexte{]}} construits à partir du contexte (partie \ref{subsec:baseentrainement}), par un algorithme de descente de gradient.
À la fin de l'algorithme, ce sont ces matrices qui donneront la représentation vectorielle des mots du vocabulaire.
Ainsi, la ligne \(i\) de la matrice \(W=\frac{W_e+W_s}{2}\) donnera la représentation du \(i\)\textsuperscript{ème} mot du vocabulaire en dimension \(dim\).

\hypertarget{subsec:baseentrainement}{%
\subsubsection{Construction de la base d'entraînement}\label{subsec:baseentrainement}}

Peu de traitements sont effectués sur la base initiale : nous mettons tout en minuscule, remplaçons les ponctuations par des espaces, mais laissons tous les chiffres et les accents.
Chaque phrase\footnote{Dans notre cas une phrase correspond à un tweet, même si ce tweet peut être composé de plusieurs phrases.}
est ensuite \emph{tokénisée} par la chaîne de caractère correspondant à un espace \texttt{"\ "} : on considère qu'il y a autant de mots de que chaînes de caractères séparées par un espace\footnote{Les mots composés sont donc considérés comme plusieurs mots distincts.}. Par exemple, la phrase :

\begin{quote}
\LARGE \textbf{``}\normalsize \emph{Que pensez-vous de CE projet?(i.e. : qu'avez-vous retenu en 10min ?)} \LARGE \textbf{''}\normalsize
\end{quote}

est décomposée en 14 mots \texttt{{[}que,\ pensez,\ vous,\ de,\ ce,\ projet,\ i,\ e,\ qu,\ avez,\ vous,\ retenu,\ en,\ 10min{]}}.

Nous effectuons enfin un traitement sur les mots rares. Si un mot apparaît strictement moins de 10 fois, nous lui affectons la valeur «~lowfrequency~»\footnote{Dans le corpus de tweets que nous utiliserons ultérieurement, un mot rare apparaît moins de 10 fois sur les 31 400 000 mots utilisés dans le corpus. Rassembler les mots rares permettra de passer d'un vocabulaire d'environ 635 000 mots à 70 000 mots en réduisant le nombre de mots de seulement 3,0 \%.}.

Comme décrit dans la partie \ref{subsec:word2vec}, les couples \texttt{{[}focus,\ contexte{]}} dépendent d'un hyperparamètre : la fenêtre \(w\). Pour éviter que les mots trop fréquents, souvent peu informatifs (comme les pronoms personnels), soient sur-entraînés, deux traitements sont effectués :

\begin{enumerate}
\def\labelenumi{\arabic{enumi}.}
\item
  Pour chaque phrase on effectue un sous-échantillonnage (\emph{subsampling}). Pour chaque mot \(w_i\) on note \(z(w_i)\) la proportion d'apparition de ce mot, c'est-à-dire le rapport entre le nombre de fois que ce mot apparaît et le nombre total de mots. La probabilité de garder un mot le mot \(w_i\) est donnée par :
  \[
  \mathbb P(w_i) = \min\left\{\left(\sqrt{\frac{z(w_i)}{q}} + 1 \right)
  \times
  \frac{q}{z(w_i)},1\right\}
  \]
  Le paramètre \(q\) appelé «~sample~» -- échantillonnage -- contrôle le nombre de mots sous-échantillonnés (plus il est grand, plus la probabilité de garder le mot \(w_i\) est grande). Si \(q\) vaut 0,001 (valeur par défaut) alors par exemples :

  \begin{itemize}
  \tightlist
  \item
    \(\mathbb P(w_i) = 1\) (\(w_i\) est toujours gardé) lorsque \(z(w_i)\leq 0,0026\), c'est-à-dire si \(w_i\) représentent moins de 0,26 \% du nombre total de mots.\\
  \item
    \(\mathbb P(w_i) = 0,5\) (50 \% de chance de garder \(w_i\)) lorsque \(z(w_i)=0,00746\).\\
  \item
    \(\mathbb P(w_i) = 0,033\) (3,3 \% chance de garder \(w_i\)) lorsque \(z(w_i)=1,0\) (si le corpus n'est constitué que du mot \(w_i\), ce qui serait bien sûr absurde).
  \end{itemize}
\end{enumerate}

Ce sous-échantillonnage est effectué de manière indépendante pour chaque phrase : un même mot peut donc être sous-échantillonné dans une phrase et ne pas l'être dans une autre.

\begin{enumerate}
\def\labelenumi{\arabic{enumi}.}
\setcounter{enumi}{1}
\tightlist
\item
  Pour chaque phrase, on tire au hasard (selon une loi uniforme) un mot focus pour lequel on tire un mot \emph{contexte} au hasard dans la fenêtre \(w\), en imposant que les deux mots choisis soient parmi les mots sous-échantillonnés\footnote{Si pour une phrase, aucun couple \texttt{{[}focus,\ contexte{]}} ne figurent simultanément dans les mots sous-échantillonnés, alors aucun couple n'est retenu pour cette phrase.}.
  Par exemple, nous supposons que dans la phrase \texttt{{[}que,\ pensez,\ vous,\ de,\ ce,\ projet,\ i,\ e,\ qu,\ avez,\ vous,\ retenu,\ en,\ 10min{]}}, les mots sous-échantillonnés sont les mots en position 2, 5, 6, 8, 9, 10, 11, 12, 13, 14.
  Pour mieux comprendre, nous remplaçons les mots non échantillonnés par «~nonsubsampled~». La phrase devient alors \texttt{{[}nonsubsampled,\ pensez,\ nonsubsampled,\ nonsubsampled,\ ce,\ projet,\ nonsubsampled,\ e,\ qu,\ avez,\ vous,\ retenu,\ en,\ 10min{]}}.
  Si \(w=2\) alors le mot focus tiré ne peut pas être «~pensez~» puisque dans ce cas il n'y aurait aucun mot contexte associé. Si le mot focus tiré est «~qu~» alors le mot contexte est tiré au hasard parmi \texttt{{[}e,\ avez,\ vous{]}}.
\end{enumerate}

Ce mécanisme va être répété sur toutes les phrases du corpus et l'ensemble du corpus va être parcouru plusieurs fois. Le nombre de fois que l'ensemble du corpus est parcouru est appelé \emph{epochs}.

\hypertarget{subsec:descentedegradient}{%
\subsubsection{Descente de gradient}\label{subsec:descentedegradient}}

Pour chaque couple \texttt{{[}focus,\ contexte{]}}, les matrice \(W_e\) et \(W_s\) sont mises à jour par descente de gradient. C'est-à-dire que \(\theta^{(t)} = W_e\) et \(\theta^{(t)} = W_s\), les matrices obtenues après la \(t\)\textsuperscript{ème} itération de l'algorithme, sont mises à jour par l'équation :
\[\theta^{(t+1)} = \theta^{(t)} - \eta \nabla_\theta Loss(\theta^{(t)})\]
avec \(\eta\) le taux d'apprentissage (un hyperparamètre à fixer) et \(L(\theta)\) la fonction de perte.

Le modèle \emph{word2vec} a initialement été construit en utilisant une fonction de perte dérivée de la fonction \emph{softmax} (voir partie \ref{subsec:softmax} et \cite{Mikolov}). L'algorithme a ensuite été amélioré en utilisant le \emph{negative sampling} (voir partie \ref{subsec:negsampling} et \cite{MikolovNS}).

\hypertarget{subsec:softmax}{%
\paragraph{Version softmax}\label{subsec:softmax}}

Soit \(w_1,\,\dots,\,w_T\) les mots utilisés pour entraîner le modèle. L'objectif du modèle Skip-Gram est, étant donné un mot focus, de prévoir quels sont les mots voisins contextes dans une certaine fenêtre \(w\). Mathématiquement, on cherche à maximiser la quantité :
\begin{equation}
\frac 1 T\sum_{t=1}^T\sum_{-w\leq j \leq w,\,j\ne 0} \log \mathbb P(w_{t+j}\vert w_{t})
\label{eq:objSoftMax}
\end{equation}
où :

\begin{itemize}
\item
  les \(w_{t+j}\) sont les mots voisins de \(w_t\) (\(w_t\) est donc un mot focus et \(w_{t+j}\) un mot contexte) ;
\item
  \(\mathbb P(w_{t+j}\vert w_{t})\) est la probabilité d'observer le mot contexte \(w_{t+j}\) sachant que l'on a observé le mot focus \(w_t\).
  Cette quantité est calculée en fonction des matrices \(W_e\) et \(W_s\) à partir de la fonction softmax
  \footnote{Étant donné le vecteur \(z=(z_1,\,\dots,\,z_n)\)) la fonction softmax est la fonction qui à \(z\) associe le vecteur dont la \(j\)\textsuperscript{ème} coordonnée est égale à \(\frac{\exp(z_j)}{\sum_{i=1}^n\exp(z_i)}\).}. En notant \(n\) la taille du vocabulaire et \(W_{e,w_i}\) et \(W_{s,w_i}\) les représentations vectorielles du mot \(w_i\) respectivement dans la matrice d'entrée et de sortie, cette probabilité est égale à\footnote{Dans tout le rapport nous utiliserons la notation \(^{t}X\) pour désigner la transposée de la matrice \(X\).} :
  \[
  \mathbb P(w_{contexte}\vert w_{focus}) = 
  \frac{
  \exp(W_{e,w_{focus}}\times {}^tW_{s,w_{contexte}})
  }{
  \sum_{i=1}^n\exp(W_{e,w_{focus}}\times {}^tW_{s,w_{i}})
  }
  \]
\end{itemize}

Maximiser l'équation \eqref{eq:objSoftMax} revient à minimiser la fonction de perte suivante pour chaque couple {[}focus, contexte{]} :

\[
Loss_{1}=-\log\mathbb P(w_{contexte}\vert w_{focus}) =
-W_{e,w_{focus}}\times {}^tW_{s,w_{contexte}}+
\log\left(\sum_{i=1}^n\exp(W_{e,w_{focus}}\times {}^t W_{s,w_i})\right)
\]
L'inconvénient de cette méthode est qu'elle est très gourmande en temps de calcul.
En effet, pour chaque couple \texttt{{[}focus,\ contexte{]}}, la complexité du calcul de \(\log\mathbb P(w_{contexte}\vert w_{focus})\) est proportionnelle à la taille du vocabulaire.
La taille du vocabulaire pouvant être très grande (par exemple, dans notre base de tweets, cette taille est de 70 330), le temps de calcul peut vite devenir très important.

C'est pourquoi la version \emph{softmax} est très peu utilisée dans les implémentations de Skip-Gram. Une approche alternative, \emph{negative sampling} avec une fonction sigmoïde, moins gourmande en temps de calcul, est alors souvent préférée\footnote{Une autre alternative à l'approche \emph{softmax} parfois utilisée est l'approche \emph{hierarchical softmax} qui se base sur l'utilisation d'arbres binaires de classification. La complexité de cet algorithme est proportionnelle à \(\log_2n\) mais reste plus importante que celle de l'approche \emph{negative sampling}.}.

\hypertarget{subsec:negsampling}{%
\paragraph{\texorpdfstring{Version \emph{negative sampling}}{Version negative sampling}}\label{subsec:negsampling}}

Le \emph{negative sampling} est basé sur le concept du \emph{Noise Contrastive Estimation} -- estimation contrastée du bruit -- où on cherche, à partir d'un modèle logistique, à différencier un vrai signal (un vrai couple \texttt{{[}focus,\ contexte{]}}) d'un faux (un bruit qui correspondrait à un faux couple \texttt{{[}focus,\ contexte{]}} généré aléatoirement).

Dans cette approche, plutôt que de mettre à jour l'ensemble des représentations vectorielles des mots pour chaque couple \texttt{{[}focus,\ contexte{]}}, on tire \(K\) mots au hasard du vocabulaire \((w_{neg,\,i})_{i=1..K}\), selon une loi \(P\) (définie plus tard), en considérant que ces mots ne seront pas des mots voisins de \texttt{focus}\footnote{Il est bien sûr possible que parmi les mots tirés au hasard il y ait des mots qui soient vraiment dans le contexte. Cependant, puisque la taille du vocabulaire est très grande, on considère que cette erreur est négligeable.}.

L'approche \emph{softmax} peut être vue comme un problème de classification multiclasses : étant donné un mot \texttt{focus}, on estime la probabilité que les autres mots soient parmi ses voisins (chaque classe étant un mot du vocabulaire).
L'idée du \emph{negative sampling} est de transformer ce problème de classification multiclasses en un problème de classification binaire d'une variable \(D\) : pour chaque couple \texttt{{[}focus,\ mot2{]}}, on cherche à déterminer si \texttt{mot2} est dans le contexte de \texttt{focus}. Si c'est le cas, alors \(D=1\) (mot2 est positif et est le \texttt{contexte}), sinon \(D=0\) (mot2 est négatif, il appartient à \((w_{neg,\,i})_{i=1..K}\)).

On cherche donc à maximiser \(\mathbb P(D=1\vert w_{focus},w_{contexte})\) et \(\mathbb P(D=0\vert w_{focus},w_{neg,\,i})\).
Pour estimer ces probabilités, on utilise une fonction sigmoïde plutôt que la fonction softmax :
\[
\mathbb P(D=1\vert w_{focus},w_{contexte})=\sigma(W_{e,w_{focus}}{}^tW_{s,w_{contexte}}) = 
\frac{1}{1+\exp(-W_{e,w_{focus}}{}^tW_{s,w_{contexte}})}
\]
et :
\[
\mathbb P(D=0\vert w_{focus},w_{neg,\,i})=\sigma(-W_{e,w_{focus}}{}^tW_{s,w_{neg,\,i}}) = 
\frac{1}{1+\exp(W_{e,w_{focus}}{}^tW_{s,w_{neg,\,i}})}
\]
Par rapport à l'approche \emph{softmax}, on cherche toujours à maximiser la quantité de l'équation \eqref{eq:objSoftMax} mais en estimant \(\log\mathbb P(w_{contexte}\vert w_{focus})\) par :
\[
\log\mathbb P(w_{contexte}\vert w_{focus}) =
\log\underbrace{\sigma (W_{e,w_{focus}}{}^tW_{s,w_{contexte}})}_{
\mathbb P(D=1\vert w_{focus},w_{contexte})
}+
\sum_{i=1}^K\mathbb E_{w_{neg,i}\sim P}[
\log
\underbrace{\sigma (-W_{e,w_{focus}}{}^tW_{s,w_{neg,\,i}})}_{
\mathbb P(D=0\vert w_{focus},w_{neg,\,i})
}
]
\]

Ainsi, pour chaque couple \texttt{{[}focus,\ contexte{]}} et un ensemble \((w_{neg,\,i})_{i=1..K}\) de mots négatifs tirés, on associe la fonction de perte suivante, à minimiser :
\[
Loss_{2}=-\log\sigma (W_{e,w_{focus}}{}^tW_{s,w_{contexte}})
-
\sum_{i=1}^K
\log
\sigma (-W_{e,w_{focus}}{}^tW_{s,w_{neg,\,i}})
\]
La complexité est ici bien plus faible que pour la fonction softmax puiqu'elle est proportionnelle à \(K\).

\cite{MikolovNS} trouvent, empiriquement, que la meilleure distribution \(P\) pour générer les mots négatifs est telle que :
\[
\mathbb P_P(w_i) = \frac{z(w_i)^{3/4}}{
\sum_{j=1}^n z(w_j)^{3/4}
}
\]

Avec \(z(w_i)\) la fréquence d'apparition du mot \(w_i\).

Ils recommandent également de prendre \(K\in\{5,\dots,20\}\) pour les petites bases de données et \(K\in\{2,\dots,5\}\) pour les grandes bases de données. Dans ce projet, nous utiliserons \(K=5\) (pour chaque couple \texttt{{[}focus,\ contexte{]}} nous tirons donc 5 mots négatifs).

\hypertarget{sec:evaluation}{%
\section{Évaluation du modèle implémenté}\label{sec:evaluation}}

Malgré l'utilisation généralisée des \emph{word embeddings}, très peu de travaux théoriques expliquent ce qui est réellement capturé par ces représentations de mots.

C'est pourquoi ce modèle est principalement évalué à l'aide de méthodes empiriques.
Les méthodes que nous avons retenues pour évaluer, dans les parties qui suivent, la qualité des vecteurs-mots obtenus sont décrites plus précisément dans l'annexe \ref{annexe:commentEvaluer}.

\hypertarget{sec:corpusFictif}{%
\subsection{Évaluation sur un corpus fictif}\label{sec:corpusFictif}}

Avant de nous attaquer au jeu de données complet décrit plus bas, nous avons évalué un premier corpus fictif afin de nous assurer de la robustesse et de la validité du modèle implémenté.

Pour construire ce corpus, nous avons associé dix couples (du type \texttt{{[}voiture,\ camion{]}}), à dix mots différents que l'on considèrera comme des mots contextes (\texttt{{[}véhicule,\ moto\textbackslash{}dots{]}}). Le corpus fictif est ainsi formé de 10 000 phrases composées chacune d'un mot d'un couple, de cinq mots du contexte et de trois mots bruits, tous tirés aléatoirement. L'objectif sera de vérifier qu'en appliquant notre modèle sur ce corpus fictif, on obtient bien des vecteurs-mots proches pour des couples \texttt{{[}focus,\ contexte{]}} et éloignés pour des couples \texttt{{[}focus,\ mots\ bruits{]}}.

Pour cela, nous avons mis en œuvre les différentes techniques d'évaluation\footnote{À l'exception de la méthode par \og jugement humain \fg{} puisque le corpus est ici créé fictivement par ordinateur sans prêter attention au réel sens des mots.}
présentées dans l'annexe \ref{annexe:commentEvaluer} sur les \emph{word-embeddings} obtenus grâce à ce corpus fictif.

\begin{table}
\begin{center}
\begin{tabular}{|c|c|c|}
    \hline
    mot & similarité cosinus \tabularnewline
    \hline
    énorme & 0,991   \tabularnewline
    taille & 0,991   \tabularnewline
    \dots & \dots    \tabularnewline
    vanille & 0,061   \tabularnewline
    salissures & 0,054   \tabularnewline
    \hline
 \end{tabular}
\captionsetup{margin=0cm,format=hang,justification=justified}
\caption{Mots les plus proches de \og grand \fg{} par similarité cosinus}\label{table:tableau_evaluation}
\end{center}
\vspace{-0.3cm}
\footnotesize
\emph{Note : Paramètres utilisés : ep = 50 / lr = 0,01 / w = 5 / dim = 10.}
\end{table}

Les résultats semblent concluants : la similarité cosinus montre bien une forte corrélation entre les mots focus et contexte du corpus initial et une faible corrélation avec les mots bruits (tableau \ref{table:tableau_evaluation}).
L'ACP et l'algorithme t-SNE permettent également de montrer graphiquement cette proximité (figure~\ref{fig:figure_evaluation}).
Les clusters apparaissent de manière plus évidente avec t-SNE.

\begin{figure}
\begin{center}
\includegraphics[width=1\textwidth]{img/figures.png}
\captionsetup{margin=0cm,format=hang,justification=justified}
\caption{Évaluation du modèle sur données fictives}\label{fig:figure_evaluation}
\end{center}
\vspace{-0.3cm}
\footnotesize
\emph{Note : Paramètres utilisés : ep = 50 / lr = 0,01 / w = 5 / dim = 10.}
\end{figure}

\hypertarget{sec:hyperparametres}{%
\subsection{Choix des meilleurs hyperparamètres pour le modèle}\label{sec:hyperparametres}}

Une fois nous être assurés de la bonne implémentation du modèle (partie \ref{sec:corpusFictif}) grâce au corpus fictif, nous nous sommes attachés à identifier les hyperparamètres les plus pertinents au regard des données dont nous disposons.

Ces données correspondent à un ensemble de 1,3 million de tweets\footnote{Ces tweets, achetés à twitter, sont la propriété de l'Inria.}
postés en France entre 2013 et 2017, supposés être représentatifs de l'ensemble de tweets nationaux publiés durant cette période.

Le modèle \emph{word2vec} version Skip-gram, décrit en partie \ref{sec:word2vec}, fait en effet intervenir un certain nombre d'hyperparamètres parmi lesquels :

\begin{itemize}
\item $ep$ : le nombre d'\og \emph{epochs} \fg{}
\item $lr$ ou $\alpha$ : le \og \emph{learning rate} \fg{}, ou taux d'apprentissage
\item $w$ (\emph{window}): la taille de la fenêtre de sélection des mots contextes
\item $dim$ : la dimension des vecteurs-mots (ou \emph{word-embeddings})
\end{itemize}

Or, la performance de nombreuses méthodes de \emph{machine learning}, dont \emph{word2vec}, dépend fortement des valeurs choisies pour ces paramètres, ces valeurs étant elles-mêmes très dépendantes des données mobilisées.

Même si les méthodes d'optimisation bayésiennes deviennent de plus en plus performantes pour optimiser la valeur de ces hyperparamètres en tenant compte de leurs interactions (\cite{Hutter}), ce choix s'effectue régulièrement de manière empirique, en testant différentes valeurs d'hyperparamètres sur les données mobilisées.
C'est l'approche que nous retenons ici.

Le package \texttt{Gensim} (\og Generate Similar \fg{}), dans lequel la méthode \emph{word2vec} est implémentée, est un des outils actuels les plus robustes et performants\footnote{Grâce à sa dépendance à \texttt{NumPy}, \texttt{Gensim} puise dans des bibliothèques de bas niveau.
  Ainsi, alors que le code de haut niveau est du Python, c'est en fait du Fortran et du C hautement optimisés qui sont utilisés, ce qui rend \texttt{Gensim} bien plus performant que \texttt{PyTorch} que nous avons utilisé pour implémenter le modèle décrit en partie \ref{sec:word2vec}.}
pour la modélisation sémantique non supervisée (\cite{Rehurek}).

Nous avons choisi de mobiliser \texttt{Gensim} dans la suite de ce rapport, en parallèle du modèle que nous avons implémenté, en raison de son temps d'exécution bien plus rapide\footnote{À titre d'exemple, alors qu'une epoch sur l'ensemble des tweets met une vingtaine d'heures à tourner pour \og notre \fg{} modèle, elle met 1 minute via \texttt{Gensim}.}. Cette rapidité d'exécution nous a permis de réaliser des tests d'hyperparamètres plus nombreux.

Pour réaliser ces tests, nous avons fait tourner le modèle \emph{word2vec} plusieurs fois en modifiant un à un les paramètres. Nous avons ensuite évalué ces différents modèles par la méthode du \og jugement humain \fg{} (partie \ref{sec:jugementHumain}) en comparant la mesure de la similarité cosinus\footnote{Nous avons également évalué les modèles en utilisant (l'inverse de) la distance euclidienne à la place de la similarité cosinus.
  L'effet des paramètres devient alors bien moins clair et la performance du modèle est inférieure, ce va dans le sens de l'utilisation plus fréquente de la méthode de la similarité cosinus dans la littérature.}
entre deux mots obtenue à partir de notre modèle à l'évaluation subjective de cette proximité par des individus.
En outre, un même modèle est lancé six fois (six \og seeds \fg{} différentes) afin de construire des intervalles de confiance de la matière décrite précédemment, en empilant les six échantillons de mesure de proximités correspondant aux six implémentations d'un même modèle\footnote{Pour chaque modèle, nous calculons les statistiques de rang des 65 paires de mots de la base de jugement humain ainsi que le rang des similarités cosinus des mots obtenus en sortie du modèle.
  Nous réalisons ces actions pour les six implémentations du même modèle et empilons les résultats obtenus.
  C'est à partir de cette base empilée de 6x65 lignes moins les données manquantes que nous calculons chaque intervalle de confiance selon la formule décrite en partie \ref{sec:jugementHumain}.}.

\hypertarget{nombre-depochs-taille-de-fenuxeatre-et-taux-dapprentissage}{%
\subsubsection{Nombre d'epochs, taille de fenêtre et taux d'apprentissage}\label{nombre-depochs-taille-de-fenuxeatre-et-taux-dapprentissage}}

Pour cette première série de tests d'hyperparamètres, nous avons fixé la dimension des \emph{word-embeddings} à 50\footnote{En réalisant les mêmes tests sur uniquement 100 000 tweets, puis en testant une dimension de \emph{word-embeddings} de 20, les effets observés et commentés ici se confirment.} et évalué l'impact du nombre d'epochs, de la taille de la fenêtre et du taux d'apprentissage (figure \ref{fig:evaluation_1}) .

\begin{figure}
\begin{center}
\includegraphics[width=1\textwidth]{img/test_parametres.png}
\captionsetup{margin=0cm,format=hang,justification=justified}
\caption{Tests d'hyperparamètres : epochs, fenêtre et taux d'apprentissage}\label{fig:evaluation_1}
\end{center}
\vspace{-0.3cm}
\footnotesize
\emph{Note : Paramètres utilisés : dim = 50\newline
Le trait horizontal correspond au coefficient de Spearman calculé sur les échantillons empilés des six modèles et la barre verticale à l'intervalle de confiance associé.}
\end{figure}

\hypertarget{le-nombre-depochs}{%
\paragraph{Le nombre d'epochs}\label{le-nombre-depochs}}

~

Le nombre d'epochs a un effet net. Passer de 10 à 100 epochs fait nettement augmenter le score de corrélation de Spearman entre données subjectives et données en sortie du modèle.

\faArrowCircleRight{} Nous retenons alors le paramètre \textbf{ep = 100}.

\hypertarget{le-taux-dapprentissage}{%
\paragraph{Le taux d'apprentissage}\label{le-taux-dapprentissage}}

~

La valeur 0,02 semble donner systématiquement de meilleurs résultats que 0,01. En réalisant davantage de tests de taux d'apprentissage en fixant les autres hyperparamètres, les différents taux d'apprentissage présentent des performances similaires\footnote{En fixant les paramètres dim = 50, ep = 100 et w = 4 (celles du modèle retenu \emph{in fine}), et en testant les taux d'apprentissage 0,005, 0,01, 0,02, 0,03 et 0,04, les valeurs moyennes des corrélations s'échelonnent entre 0,41 et 0,48, soit des valeurs proches.}.

\faArrowCircleRight{} Nous retenons alors le paramètre \textbf{lr = 0,02}.

\hypertarget{la-taille-de-la-fenuxeatre}{%
\paragraph{La taille de la fenêtre}\label{la-taille-de-la-fenuxeatre}}

~

La taille de la fenêtre ne semble pas jouer un rôle majeur, et dépend beaucoup des autres paramètres choisis.

Certains travaux (\cite{Levy2}) indiquent que, suivant la taille de fenêtre choisie, les informations capturées sont différentes. Cela pourrait expliquer la complexité de choisir la \og meilleure \fg{} taille de fenêtre. Alors que les \og grandes \fg{} fenêtres capturent des informations sur le domaine du mot (autres mots de tout type étant utilisés dans des discussions connexes), les \og petites \fg{} fenêtres saisissent davantage le mot en lui-même (ses extensions, synonymes, lui sont alors proches). La valeur de 4 représente une taille de fenêtre \og ni trop grande ni trop petite\fg{} et qui présente de bons résultats dans la plupart des tests effectués.

\faArrowCircleRight{} Nous retenons alors le paramètre \textbf{w = 4}.

\hypertarget{dimension-des-vecteurs-mots}{%
\subsubsection{Dimension des vecteurs-mots}\label{dimension-des-vecteurs-mots}}

On cherche cette fois-ci à évaluer l'effet de la dimension des \emph{word-embeddings}. Selon certains papiers (comme \cite{Pennington}), la qualité des représentations vectorielles s'améliore à mesure que l'on augmente la taille du vecteur, mais seulement jusqu'à atteindre 300 dimensions\footnote{La dimension des vecteurs doit également être adaptée à la taille du vocabulaire. Un des articles fondateurs de word2vec (\cite{Mikolov}) recommande donc d'augmenter à la fois la dimension des vecteurs et la quantité de données d'apprentissage. Par exemple, avec un vocabulaire d'une centaine de mots, il serait inefficace d'utiliser des projections en grande dimension (risque de surapprentissage).}. Après 300 dimensions, la qualité des vecteurs commence à diminuer et le temps de calcul augmente considérablement.

\begin{figure}
\begin{center}
\includegraphics[width=1\textwidth]{img/test_parametres2.png}
\captionsetup{margin=0cm,format=hang,justification=justified}
\caption{Tests d'hyperparamètres : dimension des *word-embeddings*}\label{fig:figure_dim}
\end{center}
\vspace{-0.3cm}
\footnotesize
\emph{Note : Paramètres utilisés : ep = 100 / w = 4 / lr = 0,02.\newline
Le trait horizontal correspond au coefficient de Spearman calculé sur les échantillons empilés des six modèles et la barre verticale à l'intervalle de confiance associé.}
\end{figure}

En pratique, en comparant l'effet de la dimension des vecteurs (modèle fixé à ep~=~100, w~=~4 et lr~=~0,02), on observe bien une augmentation de l'efficacité du modèle jusqu'en dimension 300 et une efficacité moindre en dimension 500 (figure~\ref{fig:figure_dim}). Bien que l'efficacité du modèle semble meilleure en dimension 300, la dimension 100 améliore la rapidité de l'algorithme, pour des résultats d'une qualité similaire.

\faArrowCircleRight{} Nous retenons alors le paramètre \textbf{dim = 100}.

\hypertarget{uxe9valuation-sur-le-corpus-final}{%
\subsection{Évaluation sur le corpus final}\label{uxe9valuation-sur-le-corpus-final}}

\hypertarget{avec-notre-moduxe8le}{%
\subsubsection{\texorpdfstring{Avec \og notre \fg{} modèle}{Avec notre  modèle}}\label{avec-notre-moduxe8le}}

Nous avons ensuite fait tourner le modèle que nous avons implémenté en utilisant les paramètres retenus précédemment\footnote{w~=~4, lr~=~0,02 et dim~=~100} mais uniquement sur 100~000 tweets et 80 epochs pour des questions de temps de calcul\footnote{Près de 18 heures.}.

Les résultats obtenus semblent relativement satisfaisants. La recherche des plus proches voisins par similarité cosinus (dont quelques exemples sont illustrés en tableau \ref{table:knn_ark}) donne des résultats proches de l'intuition.

\begin{table}
\begin{center}
\begin{tabular}{|c|c|c|c|}
    \hline
\textbf{bonjour} & \textbf{femme} & \textbf{1} & \textbf{samedi} \tabularnewline
\emph{(669 apparitions)} & \emph{(264 apparitions)} & \emph{(765 apparitions)} & \emph{(203 apparitions)} \tabularnewline
       \hline
\includegraphics[height=4mm]{img/emojis/1.png} (0,59) & quelle (0,49) & 5 (0,55) & soir (0,57) \tabularnewline
\includegraphics[height=4mm]{img/emojis/2.png} (0,59) & cette (0,46) & mois (0,51) & vivement (0,51) \tabularnewline
merci (0,54) & une (0,44) & 10 (0,49) & demain (0,50) \tabularnewline
nuit (0,48) & vie (0,44) & 2 (0,48) & end (0,48) \tabularnewline
bisous (0,47) & grippe (0,44) & top (0,48) & weekend (0,47) \tabularnewline
bonne (0,47) & belle (0,43) & depuis (0,47) & matin (0,45) \tabularnewline
\includegraphics[height=4mm]{img/emojis/3.png} (0,46) & ma (0,43) & saison (0,46) & jeudi (0,45) \tabularnewline
vous (0,46) & magnifique (0,43) & ans (0,44) & prochain (0,43) \tabularnewline
plaisir (0,44) & nouvelle (0,43) & jours (0,43) & week (0,43) \tabularnewline
allez (0,43) & vidéo (0,39) & 3 (0,43) & \includegraphics[height=4mm]{img/emojis/4.png} (0,42) \tabularnewline
    \hline
 \end{tabular}
\captionsetup{margin=0cm,format=hang,justification=justified}
\caption{10 plus proches voisins par similarité cosinus avec \og notre \fg{} modèle}\label{table:knn_ark}
\end{center}
\vspace{-0.3cm}
\footnotesize
\emph{Note : Paramètres utilisés : ep = 80 / w = 4 / lr = 0,02 / dim = 100 / base : 100 000 tweets\newline
La similarité cosinus de chaque paire de mots est renseignée entre les parenthèses.}

\end{table}

Par ailleurs, le coefficient de Spearman entre la similarité cosinus des mots obtenus et le jugement humain est de 0,571 (p-valeur : 4,1 \%). Toutefois, ce bon résultat est à considérer avec précaution puisque seuls 13 des couples de mots de la base RG-65 ont été reconnus dans le corpus de 100~000 tweets que nous utilisons ici.

Enfin, les représentations graphiques des positions des mots via des ACP et les sommes vectorielles sur les mots\footnote{Comme l'exemple de \(\overrightarrow{Paris} - \overrightarrow{France} + \overrightarrow{Italie} = \overrightarrow{Rome}\) dans \cite{Mikolov}} donnent des résultats bien moins concluants que le modèle \texttt{Gensim} entraîné sur l'ensemble des tweets (partie \ref{sec:gensimresultats}).

\hypertarget{sec:gensimresultats}{%
\subsubsection{\texorpdfstring{Avec le modèle \texttt{Gensim}}{Avec le modèle Gensim}}\label{sec:gensimresultats}}

Le modèle \texttt{Gensim}\footnote{w~=~4, lr~=~0,02, dim~=~100 et ep~=~100.} donne des résultats encore plus convaincants que précédemment, ayant été davantage entraîné, et sur un corpus plus fourni (ensemble des tweets).
En effet, les vecteurs-mots en sortie du modèle \texttt{Gensim} sur l'ensemble des tweets (figure \ref{fig:acp_freq_gensim}) sont davantage répartis dans l'ensemble du plan, alors que les mots en sortie du modèle que nous avons implémenté sur 100 000 tweets sont répartis en fonction de leur nombre d'occurrences, les mots les moins fréquents n'ayant probablement pas (ou peu) été entraînés (figure \ref{fig:acp_freq_ark}).

\begin{figure}[h]
\begin{minipage}{.5\textwidth}
  \centering
  \includegraphics[width=1\linewidth]{img/acp_freq_gensim.png}
  \captionof{figure}{Position des mots en fonction de\newline leur nombre d'occurrences (Modèle \texttt{Gensim})}
  \label{fig:acp_freq_gensim}
\end{minipage}%
\begin{minipage}{.5\textwidth}
  \centering
  \includegraphics[width=0.72\linewidth]{img/acp_freq_ark.png}
  \captionof{figure}{Position des mots en fonction de\newline leur nombre d'occurrences (\og notre \fg{} modèle)}
  \label{fig:acp_freq_ark}
\end{minipage}
\footnotesize
\emph{Note : Paramètres utilisés : ep = 100 (gauche) ou 80 (droite) / w = 4 / lr = 0,02 / dim = 100.\newline
Méthode utilisée : ACP, deux premiers axes.}
\end{figure}

Le coefficient de Spearman a une valeur semblable à précédemment : 0,495 mais sa p-valeur est proche de 0 \% et, cette fois-ci, 52 des couples de mots de la base RG-65 ont été reconnus dans le corpus de tweets.

Les 10 plus proches voisins calculés par similarité cosinus (tableau \ref{table:knn_gensim}) semblent encore davantage pertinents.
Les plus proches voisins de \og \(1\) \fg{} contiennent davantage de chiffres, de \og samedi \fg{} davantage de jours de la semaine et le tableau contient désormais des synonymes de \og femme \fg{} et de \og bonjour \fg{}.
Certains mots surprenants subsistent toutefois, comme par exemple \og transmets \fg{}, \og désagrément \fg{} et \og betembourg \fg{}, voisins de \og bonjour \fg{}.
Toutefois, la fréquence d'apparition de ces mots dans le corpus est faible (moins d'une centaine d'occurrences).
La projection de certains vecteurs-mots sur les deux premiers axes d'une ACP (figure \ref{fig:acp_gensim}) nous confirme la qualité de l'entraînement du corpus sur l'ensemble de tweets.

\begin{table}[h]
\begin{center}
\begin{tabular}{|c|c|c|c|}
    \hline
\textbf{bonjour} & \textbf{femme} & \textbf{1} & \textbf{samedi} \tabularnewline
\emph{(17 043 apparitions)} & \emph{(6 177 apparitions)} & \emph{(21 055 apparitions)} & \emph{(4 917 apparitions)} \tabularnewline
       \hline
bonsoir (0,85) & fille (0,86) & 2 (0,65) & vendredi (0,88) \tabularnewline
bjr (0,75) & copine (0,74) & 3 (0,64) & jeudi (0,86) \tabularnewline
hello (0,71) & meuf (0,71) & 6 (0,63) & lundi (0,83) \tabularnewline
salut (0,66) & demoiselle (0,66) & 4 (0,62) & mercredi (0,83) \tabularnewline
coucou (0,55) & nana (0,66) & 7 (0,60) & dimanche (0,83) \tabularnewline
transmets (0,49) & nièce (0,66) & 5 (0,58) & mardi (0,76) \tabularnewline
désagrément (0,48) & sœur (0,65) & 9 (0,58) & demain (0,72) \tabularnewline
avezvous (0,48) & barbe (0,65) & 8 (0,56) & barathon (0,56) \tabularnewline
bettembourg (0,48) & maman (0,64) & 1e (0,55) & 22h45 (0,55) \tabularnewline
hey (0,47) & princesse (0,64) & 34 (0,53) & 20h (0,54) \tabularnewline
    \hline
 \end{tabular}
\captionsetup{margin=0cm,format=hang,justification=justified}
\caption{10 plus proches voisins par similarité cosinus avec le modèle \texttt{Gensim}}\label{table:knn_gensim}
\end{center}
\vspace{-0.3cm}
\footnotesize
\emph{Note : Paramètres utilisés : ep = 100 / w = 4 / lr = 0,02 / dim = 100 / base : ensemble des tweets\newline
La similarité cosinus de chaque paire de mots est renseignée entre les parenthèses.}
\end{table}

\begin{figure}[h]
\begin{center}
\includegraphics[width=0.5\textwidth]{img/acp_gensim.png}
\captionsetup{margin=0cm,format=hang,justification=justified}
\caption{ACP sur un corpus réduit de mots}\label{fig:acp_gensim}
\end{center}
\vspace{-0.3cm}
\footnotesize
\emph{Note : Paramètres utilisés : ep = 100 / w = 4 / lr = 0,02 / dim = 100 / base : ensemble des tweets }
\end{figure}

\begin{figure}[h]
\begin{minipage}{.5\textwidth}
  \centering
  \includegraphics[width=0.92\linewidth]{img/acp_reine.png}
  \captionof{figure}{\ $\protect\overrightarrow{Roi} - \protect\overrightarrow{Homme} + \protect\overrightarrow{Femme} = $ ?}
  \label{fig:acp_reine}
\end{minipage}%
\begin{minipage}{.5\textwidth}
  \centering
  \includegraphics[width=0.85\linewidth]{img/acp_rome.png}
  \captionof{figure}{\ $\protect\overrightarrow{Paris} - \protect\overrightarrow{France} + \protect\overrightarrow{Italie} = $ ?}
  \label{fig:acp_rome}
\end{minipage}
\footnotesize
\emph{Note : Paramètres utilisés : ep = 100 / w = 4 / lr = 0,02 / dim = 100.\newline
Les mots en vert correspondent à ceux présents dans l'opération, le mot en rouge le mot que l'on serait supposé trouver et les mots en orange les 10 mots les plus proches du résultat de l'opération vectorielle.}
\end{figure}

Enfin, nous avons réalisé des opérations sur les mots-vecteurs.
Si l'opération \(\overrightarrow{Roi} - \overrightarrow{Homme} + \overrightarrow{Femme} = \overrightarrow{Reine}\) (figure \ref{fig:acp_reine}) semble fonctionner
\footnote{Les similarités cosinus obtenues sont les suivantes : \(corr(\overrightarrow{Roi}, \overrightarrow{Homme}) = 0,34\), \(corr(\overrightarrow{Homme}, \overrightarrow{Femme}) = 0,35\) et \(corr(\overrightarrow{Roi} - \overrightarrow{Homme} + \overrightarrow{Femme} , \overrightarrow{Reine}) = 0,67\). \(\overrightarrow{Reine}\) est bien le mot le plus proche de la somme vectorielle calculée.}, l'opération \(\overrightarrow{Paris} - \overrightarrow{France} + \overrightarrow{Italie}\) (figure \ref{fig:acp_rome}) n'identifie pas \og Rome \fg{} (similarité cosinus de 0,18 seulement) dans les mots les plus proches mais d'autres villes comme \og Lyon \fg{} (similarité cosinus de 0,62).
\og Rome \fg{} semble effectivement située \og trop en haut\fg{} dans le plan de l'ACP par rapport aux autres villes.
Peut-être ce mot n'a-t-il pas suffisamment été entraîné (246 apparitions dans les tweets contre 46 433 pour Lyon par exemple) pour que le vecteur-mot obtenu soit pertinent, ou peut-être que, dans les tweets mobilisés, le mot \og Rome \fg{} s'utilise dans un contexte différent de l'article de Mikolov.

\hypertarget{sec:sentimentalAnalysis}{%
\section{Construction d'un indice mensuel de sentiment moyens des tweets}\label{sec:sentimentalAnalysis}}

Afin de créer un indice mensuel de sentiment moyens des tweets, nous allons utiliser les vecteurs-mots en sortie du modèle \emph{word2vec} couplé avec une base de tweets qui concernent les transports urbains qualifiés de positifs ou négatifs.
Nous comparerons enfin cet indicateur avec indicateur synthétique de confiance des ménages (Camme, Insee). L'ensemble de ces données sont décrites dans l'encadré \ref{enc:encadre1}.

\hypertarget{pruxe9dire-le-sentiment-dun-tweet-uxe0-partir-des-word-embeddings}{%
\subsection{\texorpdfstring{Prédire le sentiment d'un tweet à partir des \emph{word-embeddings}}{Prédire le sentiment d'un tweet à partir des word-embeddings}}\label{pruxe9dire-le-sentiment-dun-tweet-uxe0-partir-des-word-embeddings}}

Nous cherchons dans cette partie à construire un modèle permettant de prédire le sentiment (\(+1\) ou \(-1\)) associé à un tweet à partir des mots qui le composent.
Nous utilisons pour cela la base annotée sur les transports (encadré \ref{enc:encadre1}).
Nous comparerons deux approches : une première basée sur les sentiments moyens de chaque mot des tweets (partie \ref{sec:sentiments}) qui servira de référence (\emph{baseline}) pour évaluer l'efficacité de la seconde approche basée sur les \emph{word-embeddings} (partie \ref{sec:wordembeddings}) .

\hypertarget{sec:sentiments}{%
\subsubsection{Prédiction à partir du sentiment moyen des mots}\label{sec:sentiments}}

Ce premier modèle de prédiction du sentiment utilise l'information des tweets labelisés pour déterminer un sentiment moyen par mot.
Le sentiment prédit d'un tweet \(t\) composé de \(n\) mots sera :
\[S_{1,\gamma}(t) = 2 \times \mathds{1}\left\{ \frac{1}{n} \sum \limits_{i=1}^n \alpha_i \geq \gamma\right\} - 1 \qquad \in [0,1]\]
avec \(\gamma \in [-1,1]\) un seuil fixé, \(\alpha_i = \frac{nb_+(i) - nb_-(i)}{nb_+(i) + nb_-(i)} \in [-1,1]\) le sentiment moyen du mot \(i\) calculé à partir du nombre de tweets positifs (\(nb_+(i)\)) et négatifs (\(nb_-(i)\)) dans lesquels il apparaît.

Afin d'évaluer l'efficacité du modèle, nous séparons le corpus de la base de tweets sur les transports en une base d'entraînement (environ 16 000 tweets) et une base de test (environ 7 000 tweets). La base d'entraînement sert à calculer les sentiments moyens des mots\footnote{13 \% des mots du vocabulaires et 1 \% des mots utilisés dans les tweets de la base test sont absents de la base d'entraînement. Pour ces mots, le sentiment moyen \(\alpha_i\) est fixé à 0.} et à déterminer le \(\gamma\) qui présente les meilleurs performances, et la base de test permet d'estimer le modèle, c'est-à-dire prédire pour chaque tweet un sentiment, que l'on compare au vrai sentiment.

\begin{encadre}[true]{Données utilisées}\label{enc:encadre1}

\small

\textbf{Tweets mensuels}

Il s'agit de la base des tweets postés en France utilisée précédemment (partie  \ref{sec:hyperparametres}) mais couvrant un horizon temporel plus large (2011-2018).
Nous disposons de 4 200 tweets pour tous les mois de la période sous-échantillonnés de manière à avoir autant de tweets pour certains jours (les 1, 5, 10, 15, 20, 25 et 28 de chaque mois) et heures (0h, 6h, 12h 15h 18h et 21h) fixés. 

\textbf{Tweets annotés sur les transports urbains}

Cette base est composée d'un ensemble d'environ 23 000 tweets qui concernent les transports urbains (trains SNCF, métros et bus de la RATP) annotés en termes de sentiment.
%Nous avons transformé cette annotation de façon à ce que si elle est égale à $+1$ alors le tweet est considéré comme positif, si elle est égale à $-1$ alors le tweet est considéré comme négatif.
Nous avons transformé cette annotation de façon à ce qu'une valeur de $+1$ est associée à un tweet considéré comme positif et $-1$ à un tweet considéré comme négatif.


\textbf{Indicateur synthétique de confiance des ménages}

L’indicateur synthétique de confiance des ménages provient de l'Enquête mensuelle de conjoncture auprès des ménages (Camme) de l’Insee.
Il décrit, en une variable unique, la composante commune de 8 soldes d’opinion qui correspondent à la différence entre les pourcentages de réponses positives et négatives sur différents sujets (niveau de vie passé et futur en France, situation financière personnelle passée et future, perspective de chômage, opportunité de faire des achats importants, capacité à épargner actuelle et dans les mois à venir).
Il est calculé par analyse factorielle statique, dont l’objectif est de résumer l’évolution concomitante de ces soldes aux évolutions très corrélées.
L’indicateur est normalisé de manière à avoir une moyenne de 100 et un écart-type de 10 et doit donc s’interpréter en évolution. 

\end{encadre}

On choisit \(\gamma\) tel que l'\emph{accuracy} sur la base d'entraînement\footnote{Dans notre base d'entraînement la proportion de tweets positifs (43 \%) est proche de la proportion de tweets négatifs (57 \%).
  Il n'y a donc pas de problème de déséquilibre de classe : maximiser l'\emph{accuracy} permet bien de maximiser la performance de notre algorithme.} -- c'est-à-dire le taux de tweets dont le sentiment est bien prédit -- soit maximale. On obtient \(\gamma^* = -0,14\) (figure \ref{fig:max_baseline}).
En évaluant le modèle sur la base de test, l'\emph{accuracy} vaut 89,1 \%. On constate une nette amélioration par rapport au cas où \(\gamma = 0\) (70,5 \%), qui correspond au seuil «~naturel~» puisqu'il se situe au centre de l'intervalle des valeurs possibles pour \(\alpha_i\).

\begin{figure}[h]
\begin{center}
\includegraphics[width=0.5\textwidth]{img/max_baseline.png}
\captionsetup{margin=0cm,format=hang,justification=justified}
\caption{Optimisation du seuil $\gamma$ pour le modèle à partir des sentiments moyens des mots}\label{fig:max_baseline}
\end{center}
\end{figure}

\hypertarget{sec:wordembeddings}{%
\subsubsection{\texorpdfstring{Prédiction à partir des \emph{word-embeddings}}{Prédiction à partir des word-embeddings}}\label{sec:wordembeddings}}

Nous nous intéressons maintenant à un modèle basé sur l'utilisation de nos \emph{word-embeddings}.
Nous souhaiterions idéalement que ce modèle présente de meilleures performances que celui présenté en partie \ref{sec:sentiments}.

Pour cela, nous utilisons un modèle de régression binaire avec comme prédicteurs chacune des 100 dimensions des vecteurs-mots.
Toutefois, comme il s'agit ici de prévoir le sentiment des tweets et non de mots, une première étape préalable est de transformer pour chaque tweet l'ensemble des \emph{word-embeddings} des mots qui le composent en une «~\emph{sentence-embedding}~» qui correspondra à la moyenne des vecteurs-mots.

Le modèle prédictif binaire est de la forme :
\[Y_i = \mathds{1}\left\{ \sum_{i = 1}^n \beta_i X_{i,j} + \varepsilon_i \geq 0 \right\} 
\quad \text{et nous prédisons} \quad  \mathbb{P}(Y_i = 1 | X_{i}) = F_{\varepsilon}\left(\sum_{i = 1}^n \beta_i X_{i,j}\right)\]
Avec :

\begin{itemize}
\item $Y_i$ le sentiment du tweet $i$ ;
\item $X_{i,1}, \dots, X_{i,n}$ des coordonnées de la \emph{sentence-embedding} du tweet $i$ ;
\item $\varepsilon_i$ le résidu de notre modèle de fonction de répartition $F_{\varepsilon}$ qui vaudra $F_{\varepsilon}(x) = \frac{1}{1 + e^{-x}}$ dans le cas d'un modèle logit et $F_{\varepsilon}(x) = \Phi(x)$ (fonction de répartition d'une loi $\mathcal{N}(0, 1)$) dans le cas d'un modèle probit. 
\end{itemize}

Puis nous appliquons le critère de classification suivant pour prédire le sentiment \(S_{2,\gamma}(t) \in \{0,1\}\) du tweet \(t\) avec le seuil \(\gamma \in [0,1]\) :
\[
\begin{cases}
S_{2,\gamma}(t) = 0 &  \text{ si} \quad \mathbb{P}(Y_i = 1 | X_{i}) < \gamma = 0,5 \\
S_{2,\gamma}(t) = 1 &  \text{ si} \quad \mathbb{P}(Y_i = 1 | X_{i}) \ge \gamma = 0,5
\end{cases}
\]

Afin de sélectionner le meilleur modèle, nous avons comparé 8 modèles comportant différentes spécifications en utilisant le critère de l'AUC (aire sous la courbe ROC\footnote{La courbe ROC permet de représenter pour l'ensemble des seuils possibles l'évolution du nombre de vrais positifs en fonction du nombre de faux positifs.
  L'AUC, l'aire sous cette courbe, est donc compris entre 0 et 1. Une valeur proche de 0,5 correspond à un modèle aléatoire. À l'inverse, une AUC proche de 1 correspond à un très bon modèle prédictif.}) par validation croisée\footnote{Nous avons réalisé une \emph{k-fold cross validation} qui est une technique permettant de limiter le surapprentissage sur la base d'entraînement. Le principe est de découper la base en \(k = 10\) échantillons puis de considérer tour à tour chaque échantillon comme base de test, en entraînant le modèle sur les \(k-1\) échantillons restants. On calcule alors l'AUC dans chaque cas, puis on en calcule la moyenne. }. Les différences spécifications testées sont les suivantes :

\begin{enumerate}
\item \textbf{L’inclusion ou non des \emph{stop words}} (mots-vides). 
Ces mots sont des mots communs qui, en général, n’apportent pas d’information cruciale dans l’analyse textuelle. Les comparaisons de modèles indiquent qu’enlever ces mots n’améliore pas les performances de prédiction et les détériorent même légèrement. Cela peut s’expliquer par le fait que certains mots-vides renseignent sur un sentiment comme par exemple le mot-vide « pas » qui pourrait contribuer à qualifier des tweets de négatifs (son sentiment moyen dans le corpus est de $-0,25$).

\item \textbf{Le traitement des mots inconnus}. La base qui a servi à entraîner le modèle des \textbf{word-embeddings} est différente de celle que l'on utilise pour la prédiction des sentiments.
Ainsi, il existe des mots pour lesquels on ne connaît pas la représentation vectorielle. 
Deux options ont été retenues pour traiter ces mots. 
La première est de retirer ces mots des tweets analysés afin qu'ils n’influent pas sur la décision du sentiment. 
La seconde est de leur attribuer la valeur du vecteur correspondant aux mots rares (vecteur « lowfrequency », voir partie \ref{subsec:baseentrainement}) en partant du principe que s'ils sont absents du corpus d'entraînement du modèle \emph{word2vec}, ce sont bien des mots très peu fréquents. 
C’est cette deuxième option qui semble donner des résultats légèrement meilleurs, ce qui peut signifier que l’on capte alors le fait que les mots peu fréquents ne sont en moyenne pas neutres (plutôt positifs ou négatifs\footnote{En utilisant le modèle logit retenu, la prévision d'une \emph{sentence-embedding} composée uniquement du mot « lowfrequency » donne $\widehat{\mathbb{P}}(Y_i = 1 | X_{lowfrequency}) = 0,87$, ce qui laisse supposer que les mots-rares ont une consonance davantage positive.}). 

\item \textbf{La modélisation probit ou logit}. En plus de comparer les AUC des différentes spécifications ci-dessus,  on compare à spécification fixée les modèles logit et probit avec des critères économétriques (AIC, BIC). C’est le modèle logit qui permet d’obtenir les mêmes résultats, quels que soient les choix faits par ailleurs concernant les mot-vides et les mots inconnus.

\end{enumerate}

Finalement, nous retenons le modèle estimé par une \textbf{régression logit} sur la base en \textbf{gardant les mots-vides} et en affectant aux mots inconnus le \textbf{vecteur des mots très peu fréquents}. L'\emph{accuracy} de ce modèle est de 69,8 \%\footnote{Cette valeur de l'\emph{accuracy} correspond au seuil \(\gamma = 0,5\). Comme dans la partie précédente, nous avons déterminé le seuil optimal sur la base de test et comme celui-ci était très proche de 0,5 (\(\gamma^* = 0,49\)), pour une \emph{accuracy} de seulement 70,0 \%, nous avons conservé le seuil médian de \(0,5\).}, valeur inférieure au modèle de référence de la partie \ref{sec:sentiments} (89,1 \%).

\hypertarget{les-limites-des-moduxe8les-utilisuxe9s}{%
\subsubsection{Les limites des modèles utilisés}\label{les-limites-des-moduxe8les-utilisuxe9s}}

La bonne performance du modèle de référence (sentiments moyens des mots par tweets) par rapport au modèle mobilisant les \emph{word-embedding} peut s'expliquer par plusieurs facteurs.

\hypertarget{les-mots-inconnus}{%
\paragraph{Les mots inconnus}\label{les-mots-inconnus}}

Une première explication est la différence entre les deux modèles en termes de mots inconnus.
Nous l'avons vu, pour le modèle de référence, 1,4 \% des mots utilisés dans les tweets de la base test des tweets les transports (13,2 \% du vocabulaire) sont absents de la base d'entraînement.
Pour le modèle basé sur les \emph{word-embedding}, il faut mécaniquement ajouter aux mots inconnus ceux pour lesquels nous ne disposons pas de vecteurs-mots.
Ainsi, les mots inconnus sont bien plus nombreux puisqu'ils représentent 4,6 \% des mots (36,2 \% du vocabulaire).

Toutefois, la présence de mots inconnus ne semble pas biaiser la prédiction du sentiment d'un tweet.
En effet, la distribution de la part des mots inconnus est étonnamment similaire selon si on se restreint aux tweets bien prédits ou à ceux mal prédits (figure \ref{fig:mots_inconnus}).

\begin{figure}[ht]
\begin{center}
\includegraphics[width=0.5\textwidth]{img/mots_inconnus.png}
\captionsetup{margin=0cm,format=hang,justification=justified}
\caption{Part des mots inconnus dans les tweets}\label{fig:mots_inconnus}
\end{center}
\end{figure}

\hypertarget{le-processus-dannotation-utilisuxe9-pour-les-tweets-sur-les-transports-urbains}{%
\paragraph{Le processus d'annotation utilisé pour les tweets sur les transports urbains}\label{le-processus-dannotation-utilisuxe9-pour-les-tweets-sur-les-transports-urbains}}

L'éventualité que la méthode de référence reproduise en partie le processus qui a été utilisé pour générer les sentiments de la base sur les transports urbains pourrait également être une explication de la surperformance du modèle de référence.
En effet, nous ne disposons pas de la méthodologie qui a permis d'annoter ces tweets et nous avons pu relever quelques incohérences dans les sentiments attribués à certains tweets. Cela peut laisser supposer que l'annotation n'a pas été purement manuelle.
Comme de nombreux autres, le tweet très vraisemblablement ironique ci-dessous a en effet été qualifié de positif.

\begin{quote}
\LARGE \textbf{``}\normalsize \emph{C'est cool de payer un abonnement de 180 euros par mois pour attendre 1h le bus.} \LARGE \textbf{''}\normalsize
\end{quote}

\hypertarget{le-domain-shift}{%
\paragraph{\texorpdfstring{Le domain-shift\footnote{Les éléments en lien avec le \emph{dataset shift} proviennent de \cite{Candela}}}{Le domain-shift}}\label{le-domain-shift}}

Une dernière explication possible pourrait être du à ce qu'on appelle le \emph{« domain shift »} ou changement de domaine. Les techniques d'apprentissage automatique sont souvent confrontées à un défi majeur : le fait que les modèles soient entraînés sur des données différentes de celles que l'on va effectivement utiliser : c'est ce que l'on appelle le \emph{dataset shift}.

Bien que des solutions comme la correction de biais de sélection des échantillons, ou encore celle des données déséquilibrées, soient étudiées depuis de nombreuses décennies dans le monde de la statistique, certains autres problèmes, comme celui du changement de domaine (\emph{domain shift}) émergent depuis plus récemment suite à l'utilisation croissante des méthodes de \emph{machine learning}.

Le changement de domaine se caractérise par un changement de la nature, du «~domaine~» des données utilisées.
Dans le cas de \emph{deep-learning} sur des données de photographie, il pourrait par exemple s'agir de traiter de corpus de photos prises par des appareil photo calibrés de manière différentes (contraste, luminosité\dots) .
Dans notre cas précis, il s'agit de la différence entre les données de tweets publiés en France entre 2013 et 2017 et les données de tweets sur les transports urbains, qui portent sur un sujet très spécifique et pour lesquels certains mots ont peut-être une interprétation spécifique en termes de sentiments.
Modéliser le «~domain shift~» implique donc d'estimer le passage d'une représentation à une autre en utilisant des informations de distribution\footnote{La correction gamma (représentation paramétrique non linéaire de l'intensité des pixels) est par exemple une manière de pouvoir traiter le «~domain shift~» lié à l'utilisation d'appareil photos différents.}.

L'idée n'est pas ici de modéliser mathématiquement\footnote{En quelque mot, pour modéliser mathématiquement un \emph{domain shift} on considérerait une variable latente «~idéale~» \(x_0\) (une base de tweets de référence), jamais observée mais qui influerait sur \(y\) (l'indice mensuel).
  Nous observons uniquement \(x\) telle \(x=F(x_0)\) avec \(F\) qui représente la transformation de la base de tweets de référence à la base de tweets réellement utilisée, qui peut varier en fonction de la base de données \(x\) utilisée.
  La distribution \(\mathbb P(y\mid x_0)\) (de l'indice mensuel sachant le jeu de tweets idéal utilisé) est considérée comme étant la même pour les deux jeux de tweets utilisée (base annotée sur les transports urbains et base de tweets postés en France entre 2013 et 2017).
  En revanche, cette distribution est modifiée si \(F\) est modifiée.} le \emph{domain shift} mais de remarquer que la base de tweets sur les transports semble en effet qualitativement être une base particulière : de part le vocabulaire spécifique qui y est employé\footnote{Les mots «~bus~» (3800 occurrences), «~métro~» (820), «~SNCF~» (781), «~retard~» (411), «~ratp~» (237), «~gare~» (195), «~chauffeur~» (160)\dots\{\} sont parmi les plus employés dans les tweets de la base test.} mais également de part le ton particulièrement ironique de nombreux tweets\footnote{Comme en témoignent ces deux tweets «~1h de retard la SNCF, vous savez pas l'amour que je vous porte~» et «~J'ai passé 20 superbes minutes dans le RER collé à des gens que je ne connaissais pas~».
  Par ailleurs parmi les 10 mots qui appartiennent le plus à des tweets mal prédits, 4 peuvent être utilisés pour manier l'ironie (rire, mdr, ptdr et mdrrr).}.

\hypertarget{test-dune-modification-de-la-source-de-la-base-de-test}{%
\paragraph{Test d'une modification de la source de la base de test}\label{test-dune-modification-de-la-source-de-la-base-de-test}}

Nous avons testé les deux modèles sur une nouvelle base de test de tweets annotés\footnote{Cette base disponible sur github \url{https://github.com/gamebusterz/French-Sentiment-Analysis-Dataset} correspond à 1,5 million de tweets initialement en anglais associés à un sentiment.
  Son désavantage est que la traduction est de mauvaise qualité et sûrement effectuée depuis un logiciel de traduction automatique.} indépendante des autres bases.
Avec cette nouvelle base, l'\emph{accuracy} du modèle basé sur les \emph{word-embedding} est cette fois-ci supérieure à celle de l'approche de référence (61,9 \% contre 55,9 \%).

Cette nouvelle base test permet certainement de corriger une partie des biais évoqués plus haut qui pouvaient expliquer la surperformance du modèle de référence par rapport au modèle utilisant les vecteurs mots~:

\begin{itemize}
\item
  La base de test de «~github~», bien qu'assez mal traduite, ne semble pas à notre connaissance avoir été annotée par des techniques d'analyse de sentiment similaires à la méthode de référence ;
\item
  Ses tweets sont nombreux et semblent traiter de sujets divers contrairement aux tweets sur les transports, plus ciblés ;
\item
  Le modèle de \emph{word-embeddings} semble mieux réagir à la présence de mots inconnus, ici plus nombreux que précédemment\footnote{52,0 \% des mots du vocabulaire et 12,7 \% des mots de la base test de github ne sont pas dans la base d'entraînement sur les transports et 52,8 \% des mots du vocabulaire et 13,0 \% des mots ne sont ni dans la base d'entraînement de tweets sur les transports ni dans le vocabulaire des \emph{word-embeddings}.
    Ainsi, les mots de la base test «~github~» qui ne sont pas dans la base d'entraînement «~transports~» sont rarement dans les vecteurs-mots de \emph{word2vec}.}, que le modèle de référence.
  Comme il présente une efficacité similaire à lorsqu'il était évalué sur une autre base de test, il semble être plus général et s'adapter à des tweets aux contenus plus variés.
\end{itemize}

\hypertarget{sentiments-des-tweets-et-enquuxeate-de-conjoncture-aupruxe8s-des-muxe9nages}{%
\subsection{Sentiments des tweets et enquête de conjoncture auprès des ménages}\label{sentiments-des-tweets-et-enquuxeate-de-conjoncture-aupruxe8s-des-muxe9nages}}

Nous construisons désormais un indicateur mensuel de sentiment des tweets postés en France entre 2011 et 2018 (encadré \ref{enc:encadre1}) en appliquant le modèle logit sur les \emph{sentence-embeddings} (partie \ref{sec:wordembeddings}) et en calculant la moyenne des sentiments prévus pour ces tweets (0 si négatifs, 1 si positifs) pour chaque mois.\\
Afin de comparer notre indicateur à l'indicateur synthétique de confiance des ménages de l'enquête Camme, plusieurs éléments sont à prendre en considération :

\begin{itemize}
\item
  L'indicateur Camme, tout comme notre indicateur de sentiment, n'ont pas de sens en niveau. On ne cherche pas à savoir si, à un certain mois, les tweets sont plutôt positifs ou négatifs, mais plutôt à analyser l'évolution de ce sentiment.
  C'est pourquoi les deux indicateurs seront centrés réduits sur la période 2011-2018.
\item
  L'indicateur synthétique de l'enquête Camme publié au mois \(m\) porte sur l'opinion des ménages au mois \(m-1\).
  Pour le comparer à notre indicateur synthétique, il est donc nécessaire de le retarder (i.e. : de le décaler d'un mois précédent).
\item
  Notre indicateur de sentiment est brut, il n'est donc pas corrigé des varations saisonnières et des jours ouvrables (CVS-CJO).
  \emph{A contrario}, l'indicateur synthétique issu de l'enquête Camme peut-être considéré comme CVS-CJO puisqu'il est construit à partir de soldes d'opinion CVS-CJO. Pour comparer notre indicateur à l'indicateur issu de Camme, deux solutions sont possibles :

  \begin{enumerate}
  \def\labelenumi{\arabic{enumi}.}
  \item
    Corriger notre indicateur de sentiment des variations saisonnières et des jours ouvrables.
    Pour cela nous avons utilisé la méthode X-12ARIMA (annexe \ref{annexe:cvscjo}).
  \item
    Utiliser un indicateur synthétique brut issu de Camme.
    Pour cela, nous avons construit un nouvel indicateur synthétique en appliquant les poids associés à chaque variable aux soldes d'opinion bruts\footnote{Pour connaître ces poids, nous avons reproduit l'analyse factorielle utilisée pour construire l'indicateur synthétique publié par l'Insee. Faire une analyse factorielle statique revient à faire une moyenne pondérée des 8 soldes d'opinion centrés-réduits.}.
  \end{enumerate}

  Lorsque cela est possible, nous privilégierons la seconde solution : supprimer la saisonnalité serait enlever une information importante pour l'analyse de l'évolution des sentiments.
  Toutefois, de nombreuses méthodes économétriques ne sont applicables que sur les séries stationnaires et donc désaisonnalisées, c'est pourquoi nous utiliserons dans ces cas la première approche.
\end{itemize}

\hypertarget{subsec:compseries}{%
\subsubsection{Comparaison entre les séries}\label{subsec:compseries}}

Le graphique \ref{fig:bslogcam} présente l'indicateur de sentiment construit à partir de notre modèle \emph{word2vec}, celui construit à partir du modèle de référence et l'indicateur synthétique brut issu de l'enquête Camme (retardé).
Même si les tendances des deux indicateurs de sentiment diffèrent de celle de l'indicateur issu de l'enquête Camme, on observe plusieurs évolutions similaires (par exemple le pic entre mars et octobre 2012).
Par ailleurs, l'indicateur construit à partir du modèle de référence parait plus bruité que l'indicateur de sentiment construit à partir de notre modèle \emph{word2vec}.
Cela peut venir du fait que, dans les tweets utilisés pour construire les indicateurs mensuels, il y a bien plus de mots inconnus au modèle de référence basé sur les tweets sur les transports urbains (20,7 \%) qu'à notre modèle \emph{word2vec} qui comporte de nombreux \emph{vecteurs-mots} (13,6 \%).

\begin{figure}[htp]
{\centering \includegraphics[width =\textwidth]{img/rmd-graphSentiments-1}}
\captionsetup{margin=0cm,format=hang,justification=justified}
\caption{Indicateur synthétique brut de confiance des ménages de l'enquête Camme (retardé) et indicateurs mensuels de sentiment construits à partir de notre modèle \emph{word2vec} et à partir de notre modèle de référence.}\label{fig:bslogcam}
\footnotesize
\emph{Note de lecture : tous les indicateurs sont centrés-réduits (de moyenne nulle et de variance unitaire) entre janvier 2011 et décembre 2018.}

\emph{L'indicateur Camme brut, calculé par les auteurs, est obtenu en utilisant les mêmes soldes d'opinion et coefficients que ceux de l'indicateur synthétique publié par l'Insee, mais en utilisant les soldes d'opinion bruts plutôt que CVS-CJO.
Cet indicateur est retardé : la valeur au mois $m$ correspond à la publication du mois $m-1$.}
\end{figure}

Afin de regarder les similitudes entre nos indicateurs de sentiment et l'indicateur Camme, nous utilisons un algorithme de déformation temporelle dynamique --- \emph{Dynamic Time Warping} (DTW) (voir par exemple \cite{dtw} pour son implémentation en \faRProject{} que nous utilisons ici).
Cet algorithme permet de calculer une distance entre deux séries temporelles robuste aux différentes amplitudes et aux décalages temporels.
Ainsi, la distance entre une série et la même série retardée sera nulle (les deux séries représentent la même information) alors que ce ne serait pas le cas si on utilisait d'autres distances, comme la distance euclidienne.
Cette méthode conclut sur le fait que la similarité entre l'indicateur Camme et notre indicateur de sentiment issu du modèle de référence est légèrement plus grande que celle entre l'indicateur Camme et l'indicateur de sentiment calculé à partir de notre modèle \emph{word2vec}.

\hypertarget{et-dans-une-optique-de-pruxe9vision}{%
\subsubsection{Et dans une optique de prévision ?}\label{et-dans-une-optique-de-pruxe9vision}}

Les deux indices de sentiments issus des tweets du mois \(m\) peuvent se rapprocher des résultats de l'enquête Camme du mois \(m+1\) et sont alors disponibles un mois plus tôt.
Nous allons donc étudier dans quelle mesure nos indicateurs permettent de prévoir les résultats de l'indicateur synthétique de Camme.

Dans cette partie, afin que les résultats économétriques soient valides, il faut que les séries soient stationnaires.
La figure \ref{fig:bslogcam} suggère la présence d'une tendance stochastique dans tous les indicateurs (confirmée par les tests de Philipps-Perron et KPSS), y compris lorsqu'ils sont désaisonnalisés.
C'est pourquoi nous différencions les séries désaisonnalisées pour les rendre stationnaires.

L'indicateur de sentiment issu du modèle \emph{word2vec} cause au sens de Granger l'indicateur synthétique issus de l'enquête Camme (p-valeur de 0,02).
C'est-à-dire que la connaissance de l'évolution de notre second indicateur de sentiment apporte de l'information pour prévoir l'évolution de l'indicateur synthétique de l'enquête Camme.
Ce n'est pas le cas pour le premier indicateur (référence) calculé uniquement à partir des sentiments des mots des tweets (p-valeur de 0,58), malgré la similitude entre ces deux modèles (partie \ref{subsec:compseries}).

Ainsi, l'exploitation des résultats de \emph{word2vec} permet, contrairement au modèle de référence, de construire un indicateur de sentiment anticipé d'un mois utile pour prévoir l'indicateur synthétique de l'enquête Camme.

\hypertarget{discussion-et-prolongements}{%
\section*{Discussion et prolongements}\label{discussion-et-prolongements}}
\addcontentsline{toc}{section}{Discussion et prolongements}

Ce travail a permis la construction d'un indice de sentiment des tweets utile dans la prévision de l'indicateur synthétique de la confiance des ménages produit par l'Insee. Toutefois, les évolutions temporelles de ces deux séries demeurent très différentes.

La première raison de leur différence est probablement leur philosophie distincte. Alors que l'indicateur Camme résume l'opinion des ménages sur des sujets très spécifiques (évolution du niveau de vie, du chômage, etc.), notre indicateur de sentiment mesure simplement la positivité / négativité des mots employés dans des tweets, sans cibler de thématique particulière. Or, l'opinion des Français évolue différemment en fonction des sujets de société concernés (figure \ref{fig:figconclu}).

\begin{figure}[htp]
{\centering \includegraphics[width =\textwidth]{img/rmd-graphConclu-1}}
\captionsetup{margin=0cm,format=hang,justification=justified}
\caption{Opinion des Français sur les sujets sociaux et indicateur annuel de sentiment construit à partir de \emph{word2vec}.}\label{fig:figconclu}
\footnotesize
\emph{Source : Baromètre d'opinion de la DREES, 2011-2018.}

\emph{Note de lecture : tous les indicateurs sont centrés-réduits (de moyenne nulle et de variance unitaire) entre janvier 2011 et décembre 2018.}
\end{figure}

L'idéal serait de disposer d'\textbf{une base de tweets traitant de sujets divers, et bien annotés}. Il faudrait mobiliser un analyseur de sentiments performant pour les tweets en français, or peu de modèles de NLP préentraînés sont disponibles en dehors de l'anglais. Quelques avancées récentes, comme CamemBERT\footnote{source}, le cousin Français de BERT\footnote{L'outil BERT de Google permet de comprendre les intentions formulées derrière les expressions naturelles des internautes en analysant les phrases et leur contexte.} entraîné sur 138GB de texte français issu du web, \ldots{}
, tout comme l'utilisation de bases d'entraînement contenant une plus grande gradation des sentiments des tweets (permettant notamment qu'ils soient catégorisés de «~neutres~»)

En amont de \emph{word2vec}, un prétraitement plus approfondi des tweets pourrait contribuer à l'amélioration de la qualité des vecteurs-mots et à la réduction des mots inconnus. Les mots mal orthographiés (lettres en trop, oubli d'espace, mauvaise ponctuation\dots) accroissent en effet considérablement la taille du vocabulaire et la rareté des mots. Cette correction doit toutefois être effectuée avec précaution : les différentes orthographes d'un même mot peuvent parfois refléter des sentiments différents.
Par exemple, «~lol~», «~LOL~» et «~loool~» sont peut-être utilisées dans des contextes différents : ironie, blagues, rires\dots L'extension \emph{fasttext}\footnote{La bibliothèque open-source \url{https://fasttext.cc/} permet de télécharger des modèles pré-entraînés dans 157 langues différentes.} (\cite{Bojanowski}) de \emph{word2vec} apporte une solution au traitement des mots rares. On représente un mot non seulement comme lui-même mais aussi grâce à des sacs de \(n\)-grammes constitutifs.
Si \(n = 3\), «~WHERE~» est représenté non seulement par \texttt{\textless{}where\textgreater{}} mais également \texttt{\textless{}wh,\ whe,\ her,\ ere,\ re\textgreater{}}.

Enfin, l'utilisation de modèles d'analyse de sentiment plus élaborés, comme les réseaux de neurones récurrents, amélioreraient peut-être également la qualité de la prévision du sentiment d'un tweet. Il est également sûrement possible de gérer avec plus de précision les mots manquants, en étudiant leur nature et l'impact de notre méthode d'imputation sur les prévisions de sentiments.

\newpage

\hypertarget{conclusion}{%
\section*{Conclusion}\label{conclusion}}
\addcontentsline{toc}{section}{Conclusion}

Ce projet très riche nous a permis de partir à la découverte des méthodes d'apprentissage par réseaux de neurones, via le modèle \emph{word2vec}.
Nous nous sommes imprégnés de son fonctionnement et l'avons implémenté dans son ensemble grâce à la librairie \texttt{Pytorch} de Python (partie \ref{sec:word2vec}).
Au-delà de la compréhension et de l'implémentation du modèle, nous nous sommes également initiés aux tests d'hyperparamètres et à son évaluation sur un corpus fictif (partie \ref{sec:evaluation}) grâce à plusieurs méthodes (calculs de similarités cosinus, opérations vectorielles sur les mots, méthodes de réduction de dimension ACP et T-SNE et jugement humain).
Il a été fascinant d'observer à quel point le modèle présente d'excellents résultats en termes de capture sémantiques des mots dans un texte.

Dans la partie \ref{sec:sentimentalAnalysis} dédiée à l'analyse de sentiment, nous avons pu appliquer sur un cas concret plusieurs méthodes étudiées durant notre deuxième année à l'ENSAE (modèles de prédiction binaire, analyse de séries temporelles, minimisation du risque empirique / validation croisée\dots).
L'entraînement et le test du modèle logit dont les prédicteurs correspondent aux dimensions des \emph{word-embedding} nous a permis d'observer qu'en plus de représenter la proximité entre mots, le modèle \emph{word2vec} permet de capter dans une certaine mesure le sentiment de phrases.

Bien sûr, la comparaison de l'indice mensuel de sentiment moyens des tweets que nous avons construit avec l'indicateur synthétique de confiance des ménages peut à première vue paraître «~décevante~» en raison des différences observées entre les deux séries.
Toutefois, le test de causalité au sens de Granger nous indique que l'indice obtenu s'avère utile pour prédire l'indicateur Camme.

La possibilité d'obtenir deux indicateurs très proches était assez utopique en raison de leurs philosophies différentes.
Au-delà de la positivité ou négativité d'un tweet, il demeure important de déceler les sujets sur lesquels les tweets portent pour créer un indice plus ciblé.
Par ailleurs, les bases d'entraînement et de test utilisées pour entraîner le modèle comportent de nombreuses limites (\emph{domain-shift}, processus d'annotation, mots inconnus).

Les méthodes d'apprentissage étant en perpétuelle évolution, le modèle \emph{word2vec} mis en exergue en 2013, connaît déjà des «~concurrents~». Le plus connu est certainement \emph{GloVe} pour «~\emph{Global Vectors}~» (\cite{Pennington}).
Alors que \emph{word2vec} privilégie l'utilisation de «~n-grammes~» (avec l'utilisation des mots contextes qui se situent autour d'une fenêtre du mot focus), \emph{GloVe} se base sur l'ensemble les statistiques d'occurrence des mots du corpus (matrice de cooccurrence mot-mot) et prétend permettre alors, par construction, de capturer des statistiques plus générales liées au corpus.

\newpage

\hypertarget{appendix-appendix}{%
\appendix}


\hypertarget{annexe:commentEvaluer}{%
\section{Comment évaluer le modèle ?}\label{annexe:commentEvaluer}}

\hypertarget{distance-entre-deux-mots}{%
\subsection{Distance entre deux mots}\label{distance-entre-deux-mots}}

L'un des enjeux principaux du modèle étant de pouvoir estimer la proximité entre deux vecteurs-mots, nous pouvons tout d'abord mesurer cette dernière par des calculs de distance.

Il existe différents types de distances. Chacune d'elles possède des propriétés intéressantes et s'adaptent plus ou moins bien au problème traité. Nous avons ici retenu deux distances classiquement utilisées :

\begin{itemize}
\tightlist
\item
  \textbf{la distance euclidienne} : \(d_{e}(\vec{u},\vec{v}) = \left\| \vec{u} - \vec{v} \right\|_2\)\\
  Un problème est que la longueur du vecteur mot, captée dans le cas de la distance euclidienne, est positivement corrélée à la fréquence d'apparition du mot (\cite{Schakel}). Cette information peut s'avérer utile dans l'analyse de la signification des mots, notamment lorsque l'on effectue des opérations sur les vecteurs (comme l'exemple de \(\overrightarrow{Paris} - \overrightarrow{France} + \overrightarrow{Italie} = \overrightarrow{Rome}\) dans \cite{Mikolov}).\\
  Toutefois, cette dépendance à la fréquence d'apparition peut également fausser l'analyse. C'est pourquoi nous avons choisi, par la suite, de normaliser les vecteurs :
  \[ d_{e}(\vec{u},\vec{v}) = \left\| \frac{\vec{u}}{\left\| \vec{u} \right\|_2} - \frac{\vec{v}}{\left\| \vec{v} \right\|_2}  \right\|_2\]
\item
  \textbf{la similarité cosinus} : \(d_{c}(\vec{u}, \vec{v}) = \frac{\vec{u}.\vec{v}}{\left\| \vec{u} \right\|_2 \left\| \vec{v} \right\|_2 }\).\\
  La similarité cosinus correspond au produit scalaire entre les deux vecteurs normalisés. Elle mesure ainsi l'angle formé entre deux vecteurs-mots.\\
  C'est la distance que de nombreux papiers fondateurs de la méthode \emph{word2vec} (comme \cite{Mikolov} ou \cite{Levy}) utilisent, avec l'argument selon lequel les mots apparaissant dans des contextes similaires sont groupés dans la même direction durant l'entraînement.
  Une similarité est proche de \(+1\) si deux mots sont positivement reliés (proches), de \(-1\) s'ils sont négativement reliés (éloignés) et de 0 s'ils ne sont pas \og reliés \fg{}.\\
  Il est toutefois délicat d'interpréter une similarité proche de \(-1\). On pourrait intuitivement penser à des antonymes, comme \og grand \fg{} et \og petit \fg{}, mais en pratique, les antonymes sont susceptibles d'apparaître dans des contextes semblables et sont donc bien souvent positivement corrélés.
\end{itemize}

\hypertarget{analyse-en-composantes-principales}{%
\subsection{Analyse en Composantes Principales}\label{analyse-en-composantes-principales}}

Une fois le modèle \emph{word2vec} entraîné, nous obtenons des \emph{word-embeddings} pour chacun de nos mots, représentés par des vecteurs de grandes dimensions (20, 50 ou même supérieures à 100).

Dès lors, il devient complexe de bien observer la proximité entre deux mots.
C'est pourquoi il devient utile de mobiliser des méthodes de réduction de dimensions comme l'analyse en composantes principales (ACP).
En effet, l'objectif premier de cette méthode est de projeter un nuage de points sur un espace de dimension inférieure.
Cela permet de rendre l'information moins redondante et plus visuelle, tout en étant le plus proche possible de la réalité.

Considérons le cas où nous disposons de \(n\) individus (dans notre cas les mots) et de \(p\) variables (dans notre cas, leurs composantes ou dimensions issues du modèle \emph{word2vec}).
On note \(X = (x_{ij})\) la matrice de taille \((n,p)\) des données brutes, où \(x_{ij}\) représente la valeur de la \(j\)-ème variable pour le \(i\)-ème individu. Mathématiquement, pour définir l'ACP, on définit deux espaces :

\begin{itemize}
\item
  L'\emph{espace des individus}, de dimension \(p\), auquel on associe la métrique \(M\) utilisée pour le produit scalaire. Dans la suite nous utiliserons \(M =I_p\) la matrice identité. La norme et le produit scalaire associés à \(M\) sont donc euclidiens.
\item
  L'\emph{espace des variables}, de dimension \(n\), auquel associe la métrique \(N=diag(p_1,...,p_n)\) avec \(\sum_{i=1}^np_i=1\). La matrice \(N\) représente le poids donné à chaque individu. Par simplification nous utiliserons ici des poids uniformes : \(N=\frac{1}{n}I_n\). Afin de donner le même poids à toutes les variables, chaque variable est centrée-réduite : cela revient à centrer-réduire les colonnes de notre matrice \(X\). Nous notons \(Z =\bar X= (z_{ij})\) la matrice des données centrées et réduites\footnote{Nous travaillons ici dans le cadre d'une ACP normée où la matrice \(X\) a été centrée puis réduite. La réduction de \(X\) a modifié les distances initiales entre individus (\(\langle z_i,z_{i'}\rangle_M \neq \langle x_i,x_{i'}\rangle_M\)). Cela n'aurait pas été le cas si la matrice \(X\) avait été uniquement centrée (ACP non normée).}.
\end{itemize}

Pour toute métrique \(D\) (\(D=N\) ou \(D=M\)), on associe le produit scalaire \(\langle x,y\rangle_{D} = {}^t\!xD y\).
La construction des axes de l'ACP est faite par projection orthogonale.
La projection orthogonale d'un individu \(i\) (vecteur ligne) \(z_i\) sur une droite de vecteur directeur unitaire \(v\) vaut \(\langle {}^tz_i,v\rangle_{M}=z_i\times v\) et les coordonnées de projection des \(n\) individus valent \(Zv\).

Les vecteurs directeurs des axes sont définis de manière à maximiser la dispersion du nuage (son inertie\footnote{La dispersion d'un nuage de points unidimensionnel par rapport à sa moyenne se mesure par la variance.
  Dans le cadre multidimensionnel, la dispersion du nuage par rapport à son barycentre \(\bar z\) se mesure par l'inertie, qui généralise la variance.}) des individus projetés et conserver ainsi au mieux les distances entre les individus. L'inertie se définit alors comme :

\begin{align*}
I(Z) &= \sum_{i = 1}^n p_i\|z_i-\bar{z}\|_M^2 \text{ avec }
  \bar{z} = 
  \begin{pmatrix}\bar z_{1} \\
    \vdots \\ \bar z_{p}
  \end{pmatrix} =
  \begin{pmatrix}\frac 1 n \sum_{i=1}^n z_{i,1} \\
    \vdots \\ \frac 1 n \sum_{i=1}^n z_{i,p}
  \end{pmatrix} (= 0_{\mathbb R^p}\text{ dans notre cas})
\\&=\sum_{i = 1}^n \frac 1 n \sum_{j=1}^p (z_{i,j} -  \bar{z}_j)^2  \text{ car }M=I_p 
\\&=\sum_{j = 1}^p \frac 1 n \sum_{i=1}^n (z_{i,j} -  \bar{z}_j)^2
\\&=\sum_{j = 1}^p Var(z^j)\text{, avec } z^j = 
  \begin{pmatrix} z_{1,j} \\ \vdots \\  z_{n,j} 
  \end{pmatrix}
\\ &= p \text{ car les variables sont réduites}
\end{align*}

On trouve tout d'abord le vecteur directeur \(v_1\) qui orientera le premier axe de l'ACP grâce au programme suivant :
\[
v_1 =\underset{\| v \|_M = 1}{\mathrm{argmax~}} 
\underbrace{\|Zv\|_N}_{=Var(Zv)} =\underset{\| v \|_M = 1}{\mathrm{argmax~}} ^t\!vR v 
\]
où \(R = Var(Z) = \frac{1}{n} ^t\!Z Z\) est la matrice des corrélations entre les \(p\) variables.

Puis, on choisit \(v_2\) orthogonal à \(v_1\) tel que l'inertie soit toujours maximisée :
\[
v_2 =\underset{ \| v \|_M = 1,\,v \perp v_1}{\mathrm{argmax}}\;  Var(Zv)
\]
En procédant de manière séquentielle, on obtient \(q < r\) axes orthogonaux avec \(r = rg(Z)\) et \(q\) choisi par le statisticien\footnote{Différentes méthodes existent afin de déterminer le \(q\) optimal, comme la règle de Kaiser ou encore celle du coude.}.

On peut montrer que \(\forall k < q\) :

\begin{itemize}
\tightlist
\item
  \(v_k\) est un vecteur propre associé à la k\ieme{} valeur propre \(\lambda_k\) de \(R\) (les valeurs propres étant rangées par ordre décroissant) ;
\item
  la composante principale \(Zv_k\) est centrée et \(V(Zv_k)= \lambda_k\) ;
\item
  les \(Zv_k\) ne sont pas corrélés entre eux.
\end{itemize}

On obtient alors la matrice \(F = ZV\) des nouvelles coordonnées factorielles des individus, avec \(V = (v_1,\dots,v_q)\) la matrice des vecteurs propres.
Nous utilisons ici l'ACP en vue d'identifier les individus (ici, nos mots) qui sont proches. Pour ce faire, il suffit de représenter les coordonnées factorielles de la matrice \(F\) dans des repères, en général en 2 dimensions pour une question de lisibilité. Deux mots apparaissant dans des contextes similaires seront proches sur ce repère et orientés dans la même direction.

\hypertarget{algorithme-t-distributed-stochastic-neighbor-embedding}{%
\subsection{\texorpdfstring{Algorithme \emph{t-distributed Stochastic Neighbor Embedding}}{Algorithme t-distributed Stochastic Neighbor Embedding}}\label{algorithme-t-distributed-stochastic-neighbor-embedding}}

Bien que l'ACP soit une première manière de résumer l'information contenue dans nos vecteurs, elle présente des limites, notamment dans les vecteurs aux trop grandes dimensions, pour lesquels l'inertie des premiers axes de l'ACP peut se révéler faible.

Pour combler ces lacunes, un autre algorithme de réduction de dimension peut être utilisé, celui dit du \emph{t-distributed Stochastic Neighbor Embedding} (t-SNE).
Contrairement à l'ACP, cet algorithme est stochastique et non-linéaire et il favorise l'apparition de groupes de mots proches.
Sa philosophie demeure cependant identique : représenter dans un espace à dimension réduite notre nuage de points de manière à repérer les mots proches.

La première étape de l'algorithme consiste à calculer les similarités entre les \(n\) vecteurs-mots \((x_i)_{i=1...n}\). La similarité entre \(x_i\) et \(x_j\) se mesure comme étant la probabilité conditionnelle \(p_{j|i}\) de choisir \(x_j\) comme voisin de \(x_i\), si les voisins étaient tirés au sort selon une loi \(\mathcal{N}(x_i, \sigma_i)\)\footnote{\(\sigma_i\) doit être calculé de manière à adapter la loi conditionnelle aux données.
  Une faible dispersion autour de \(x_i\) entraînera un \(\sigma_i\) faible et réciproquement.
  Il s'agit de trouver le \(\sigma_i\) qui minimise ce qui est appelé en théorie de l'information la \og perplexité \fg{}, c'est-à-dire un indicateur qui décrit à quel point une distribution de probabilité réussit à prédire un échantillon.} :

\[ p_{j|i} = \frac{
\exp\left(-\frac{(d_e(x_i - x_j))^2}{2\sigma_i^2}\right)
}{
\sum_{k \neq i}
\exp\left(-\frac{(d_e(x_i - x_k))^2}{2\sigma_i^2}\right)
}\]

La seconde étape de l'algorithme consiste à trouver le nouvel espace de projection à faible nombre de dimensions.
On appellera \(g_i\) les \(x_i\) projetés dans cet espace que l'on cherche à déterminer.
On calcule maintenant les probabilité conditionnelles \(q_{j|i}\) de choisir \(g_j\) comme voisin de \(g_i\) en supposant que les \((g_i)_i\) suivent cette fois-ci une distribution de \emph{Student} -- d'où le nom de l'algorithme -- plutôt qu'une loi gaussienne\footnote{Dans un espace à faible dimension, la dispersion des vecteurs est réduite.
  La distribution de Student possède des queues plus épaisses que la loi normale, ce qui permet de mieux différencier les vecteurs distants des vecteurs similaires.}.

\[ q_{j|i} = \frac{(1+ (d_e(g_i - g_j))^2)^{-1}}{\sum_{k \neq i}{(1+ (d_e(g_i - g_k))^2)^{-1}}}\]

Afin d'obtenir les \(g_i\), on minimise, par descente de gradient, la divergence de Kullback--Leibler entre les distributions de probabilité P et Q des \(p_{ij}\) et \(q_{ij}\) définis par :
\[KL(P,Q) = \sum_{i \neq j} { p_{ij} \log{\frac{p_{ij}}{q_{ij}}}} \qquad\text{avec}\qquad p_{ij} = \frac{p_{i|j} + p_{j|i}}{2n}\]

Comme dans l'algorithme de l'ACP, l'algorithme de t-SNE nous permet d'obtenir une nouvelle projection des \(x_i\).
Il faut cependant analyser avec précaution ses résultats. L'algorithme n'étant pas linéaire, l'interprétation de la taille des \emph{clusters} obtenus ou de la distance qui les sépare n'est alors pas directe.

\hypertarget{sec:jugementHumain}{%
\subsubsection{Jugement humain}\label{sec:jugementHumain}}

Les \emph{word-embeddings} obtenus par \emph{word2vec} sont censés regrouper les mots qui apparaissent dans un contexte similaire.
Une dernière façon de vérifier la qualité de nos vecteurs-mots est de les comparer à un jugement humain. Pour ce faire, nous utilisons la liste de référence RG-65 pour le français\footnote{Le RG-65 a fait appel à 18 évaluateurs humains. La base, initialement mobilisée dans un article anglophone (\cite{Rubenstein}) a été traduite de l'anglais.} (\cite{Boumedyen}).
Elle contient 65 paires de noms communs (tableau \ref{table:human_judgement}) évaluées sur une échelle de 0 (non liés) à 4 (très liés).

\begin{table}
\begin{center}
\begin{tabular}{|c|c|c|}
    \hline
    mot 1 & mot 2 & similarité  \tabularnewline
    \hline
    corde & sourire & 0,00   \tabularnewline
    midi & ficelle & 0,00   \tabularnewline
    \dots & \dots & \dots   \tabularnewline
    corde & ficelle & 3,33   \tabularnewline
    \dots & \dots & \dots   \tabularnewline
    automobile & auto & 3,94   \tabularnewline
    coq & coq & 4,00   \tabularnewline
    \hline
 \end{tabular}
\captionsetup{margin=0cm,format=hang,justification=justified}
\caption{Base de données de jugement humain}\label{table:human_judgement}
\end{center}
\end{table}

Nous calculons ensuite la corrélation de Spearman entre les similarités cosinus de ces différentes paires issues de notre modèle (notées ici \((X_i)_{i=1..n}\)) et les scores proposés ci-dessus par des êtres humains (notés ici \((Y_i)_{i=1..n}\)).

La corrélation de Spearman est égale au coefficient de corrélation de Pearson calculé sur les variables de rang.
\[
r_s = \mathrm{corr}(\mathrm{rg}_X, \mathrm{rg}_Y) = 
\frac{\mathrm{cov}(\mathrm{rg}_X, \mathrm{rg}_Y)}{
\sigma_{\mathrm{rg}_X} \sigma_{\mathrm{rg}_Y}
}
\]
La variable de rang \(\mathrm{rg}_{X_i}\) est définie telle que \(\mathrm{rg}_{X_i}=j \iff X_i = X_{(j)}\) (\(X_i\) est la \(j\)ème plus petite variable).

Pour tester la significativité de ce coefficient, nous utilisons la loi sous \((H_0)\) de la statistique de test \(z = \arctanh(r_s) = \frac{1}{2} \ln\frac{1+r}{1-r} \overset{H_0}{\sim}\mathcal{N}(0, \frac{1}{n-3})\) et obtenons l'intervalle de confiance suivant :

\[
IC_\alpha (r_s) = \left[\tanh\left(z-\frac{q_{1-\frac{\alpha}{2}}}{\sqrt{n-3}}\right),
\tanh\left(z+\frac{q_{1-\frac{\alpha}{2}}}{\sqrt{n-3}}\right)\right]
\]
avec \(q_{1-\frac{\alpha}{2}}\) le quantile d'ordre \(1-\frac{\alpha}{2}\) d'une loi \(\mathcal{N}(0, 1)\).

\newpage

\hypertarget{annexe:cvscjo}{%
\section{Correction des effets saisonniers et jours ouvrables}\label{annexe:cvscjo}}

La figure \ref{fig:cvscjo} compare les séries brutes et corrigées des variations saisonniers et des jours ouvrables (CVS-CJO) des trois indicateurs étudiés.
Même si les séries brutes semblent très proches des séries désaisonnalisées, la présence de saisonnalité est confirmée par le F-test proposé par \cite{lytras}.

Dans cette annexe nous présentons de quelle manière nous désaisonnalisons nos deux indicateurs de sentiment calculés à partir des tweets.
Pour rappel, l'indicateur synthétique de confiance des ménages publié par l'Insee grâce à l'enquête Camme est calculé à partir de soldes d'opinion CVS-CJO : on considère donc qu'il est CVS-CJO.
L'indicateur brut associé correspond à l'indicateur que l'on obtient en utilisant les mêmes soldes d'opinion et les mêmes coefficients que dans l'indicateur publié par l'Insee mais en utilisant les soldes d'opinion bruts plutôt que CVS-CJO.

\begin{figure}[htp]
{\centering\includegraphics[width = \textwidth]{img/rmd-graphCVS-1}}
\captionsetup{margin=0cm,format=hang,justification=justified}
\caption{Comparaison des séries brutes et CVS-CJO de l'indicateur synthétique de confiance des ménages de l'enquête Camme et indicateurs mensuels de sentiment construits à partir de notre modèle \emph{word2vec} et à partir de notre modèle de référence.}\label{fig:cvscjo}
\footnotesize

\emph{L'indicateur Camme brut correspond à l'indicateur que l'on obtient en utilisant les mêmes soldes d'opinion et les mêmes coefficients que dans l'indicateur synthétique publié par l'Insee mais en utilisant les soldes d'opinion bruts plutôt que CVS-CJO.}

\emph{Les autres indicateurs ont été corrigées des variations saisonnières et des jours ouvrables (CVS-CJO) par la méthode X-12ARIMA.}
\end{figure}

Pour désaisonnaliser nos deux indicateurs de sentiment, nous avons utilisé la méthode X-12ARIMA qui fonctionne en deux étapes :

\begin{enumerate}
\def\labelenumi{\arabic{enumi}.}
\item
  La série initiale est pré-ajustées de certains effets déterministes (effets jours ouvrables et points atypiques) grâce à un modèle de régression linéaire dont les résidus suivent un modèle ARIMA (partie \ref{sec:cjo}).
\item
  Cette série pré-ajustée est ensuite désaisonnalisée par une méthode non-paramétrique qui repose sur l'usage de moyennes mobiles (partie \ref{sec:cvs}).
\end{enumerate}

\hypertarget{sec:cjo}{%
\subsection{Correction des effets jours ouvrables}\label{sec:cjo}}

Dans la correction des effets saisonniers et des effets jours ouvrables (CVS-CJO) on distingue trois types d'effets :

\begin{itemize}
\item
  un effet \emph{nombre de jours} lié au nombre de jours dans le mois
\item
  un effet \emph{type de jour} lié au nombre de jours de chaque type (lundi, mardi\dots)
\item
  un effet \emph{fêtes mobiles} lié à la variation d'une année sur l'autre de la date de certaines fêtes comme Pâques.
\end{itemize}

Dans notre cas, puisque les tweets sont tirés à des jours fixes il n'y a pas d'effet nombre de jours et l'effet fête mobile n'aurait pas de sens.
En revanche, il pourrait y avoir un effet type de jour : on peut par exemple supposer que les thèmes traités par les tweets, et donc les sentiments, sont différents en fonction du jour de la semaine.
Les sentiments exprimés par les tweets pourraient donc être influencés par les types de jours pendants lesquels les tweets sont tirés.

Pour estimer l'effet jour ouvrable, nous utilons l'approche de la méthode X-12ARIMA, notamment décrite dans \cite{L2018}.

Supposons que le \(j\)\textsuperscript{ème} jour de la semaine a un effet \(\alpha_j\) où \(j=1\) désigne le lundi, \(j=2\) le mardi,\dots, et \(j=7\) le dimanche.
Chaque \(\alpha_j\) représente par exemple le sentiment moyen d'un jour \(j\).
Si \(N_{jt}\) représente le nombre de jours \(j\) dans le mois \(t\), la longueur du mois \(t\) est alors \(N_t = \sum_{j=1}^{7} N_{jt}\) et l'effet cumulé pour ce mois (le sentiment moyen du mois) sera :
\[
TD_t = \sum_{j=1}^{7} \alpha_j N_{jt}
\]

Une première idée pour détecter et évaluer les effets de jours ouvrables dans une série est d'expliquer les valeurs de la série par les 7 variables \(N_{jt}\).
Mais ces régresseurs sont par nature saisonniers (il y a en moyenne plus de lundis en janvier qu'en février) et fortement corrélés.

Une formulation différente mais équivalente de l'effet jours ouvrables permet de résoudre en grande partie ces problèmes.
L'effet journalier moyen, les ventes moyennes d'une journée, s'écrit \(\bar{\alpha} = \sum_{j=1}^7 \alpha_j /7\).
Comme par construction \(\sum_{j=1}^7 \left(\alpha_j-\bar{\alpha}\right) = 0\), on peut écrire :
\begin{eqnarray*}
\sum_{j=1}^7 \alpha_j N_{jt} & = & \bar{\alpha}N_t + \sum_{j=1}^7 \left(\alpha_j-\bar{\alpha}\right) N_{jt} \nonumber \\
& = &  \bar{\alpha}N_t + \sum_{j=1}^6 \left(\alpha_j-\bar{\alpha}\right) \left(N_{jt} - N_{7t}\right).
\end{eqnarray*}
Ainsi, l'effet cumulatif du mois se décompose en un effet directement lié à la longueur du mois et un effet net de chaque jour de la semaine.
Comme la quantité \(\bar{\alpha}N_t\) est par nature saisonnière (janvier a toujours plus de jours que février), on utilise en fait l'égalité : \(\bar{\alpha}N_t = \bar{\alpha}N_t^* + \bar{\alpha}\left(N_t-N_t^*\right)\), où \(N_t^*\)
représente la moyenne de la longueur du mois \(t\).
En d'autres termes, \(N_t^*\) est égal à \(30\) ou \(31\) si le mois considéré n'est pas un mois de février et à \(28,25\) dans le cas contraire.
Le second terme de l'égalité est donc nul sauf pour le mois de février.
L'utilisation des variables contrastes permet donc de désaisonnaliser les régresseurs jours ouvrables.

La version actuelle de X-12ARIMA utilisent le modèle Reg-ARIMA suivant pour estimer les effets de jours ouvrables :
\begin{equation}
y_t=\beta_0 LY_t + \sum_{j=1}^{6} \beta_j \left(N_{jt} - N_{7t}\right) + \varepsilon_t
    \label{eq:eqcjotd}
\end{equation}
où \(\varepsilon_t\) suit un modèle ARIMA et \(LY_t\) désigne le régresseur « année bissextile » --- \emph{Leap Year} (LY) --- égal à:
\[
LY_{t} = \left\{ \begin{array}{rl} 
                0,75 & \mbox{si } t \mbox{ est un mois de février bissextil } \\
                -0,25 & \mbox{si } t \mbox{ est un mois de février non bissextil } \\
                0 & \mbox{sinon}
               \end{array}
         \right.
\]

Une spécification alternative consiste à regrouper les régresseurs du lundi au vendredi et ceux du samedi et du dimanche\footnote{On a donc \(\beta_1=\beta_2=\dots=\beta_5\) et \(\beta_6 = 0\).}.
On considère ainsi que l'effet d'un lundi lundi est le même que l'effet d'un mardi, etc. jusqu'au vendredi et que cet effet est différent le samedi et le dimanche. Le modèle \eqref{eq:eqcjotd} devient :
\begin{eqnarray}
y_t=\beta_0 LY_t + \tilde\beta_1 \left(\sum_{j=1}^{5}N_{jt} - (N_{6t} + N_{7t})\right) + \varepsilon'_t
\label{eq:eqcjowd}
\end{eqnarray}

Dans notre cas, puisqu'il n'y a pas d'effet longueur du mois nous omettant le régresseur \(LY_t\) dans \eqref{eq:eqcjotd} et \eqref{eq:eqcjowd}. On considère qu'il y a un effet jours ouvrables significatif lorsque \(\exists i\in \{1,2,\dots,6\}\::\:\beta_i\ne0\) ou lorsque \(\tilde\beta_1\ne0\).

Pour nos indicateurs de sentiment (modèle logistique à partir de \emph{word2vec} ou modèle de référence construit à partir du sentiment moyen des mots), nous ne détectons pas d'effet jours ouvrables.
Cela ne signifie pas pour autant que le type de jour n'a pas d'effet sur nos indicateurs de sentiment, mais que nous n'arrivons pas à en estimer un.
Cela peut venir de la méthode de collecte : tirer les tweets tous les 1, 5, 10, 15, 20, 25 et 28 de chaque mois revient à tirer tous les jours de la semaine sauf 1\footnote{Si le 1\textsuperscript{er} jour du mois est un lundi alors, parmi les 4 200 tweets, 1 200 sont tirés un lundi, 0 le mardi et 600 les autres jours de la semaine.}.
On peut donc difficilement isoler l'effet du type de jour : il faudrait pour cela faire une analyse plus approfondie en étudiant construisant un indicateur de sentiment par jour de la semaine.

\hypertarget{sec:cvs}{%
\subsection{Correction des effets saisonniers}\label{sec:cvs}}

L'objectif de la désaisonnalisation est de décomposer une série \(X_t\) en trois composantes inobservées :

\begin{itemize}
\item
  \(TC_t\) la tendance-cycle : combinaison de la tendance et du cycle ;
\item
  \(S_t\) la saisonnalité : une fluctuation qui qui se répète d'année en année à la même période ;
\item
  \(I_t\) l'irrégulier : une fluctuation résiduelle non comprise dans les deux composantes précédentes (chocs ponctuels, etc.).
\end{itemize}

Ces composantes sont combinées entre elles selon un \emph{schéma de composition} dont les plus classiques sont :

\begin{itemize}
\item
  le schéma additif : \(X_t=TC_t+S_t+I_t\) ;
\item
  le schéma multiplicatif \(X_t=TC_t\times S_t\times I_t\).
\end{itemize}

À partir de l'estimation de ces composantes, il est possible de désaisonnaliser les séries en enlevant la composante saisonnière \(S_t\).

Pour effectuer la désaisonnalisation, la méthode X-12ARIMA utilise l'algorithme X-11.
En quelques mots, c'est un algorithme qui repose sur l'utilisation de moyennes mobiles afin d'estimer itérativement les différentes composantes.

Le graphique \ref{fig:coefcvscjo} trace l'évolution des coefficients saisonniers pour nos indicateurs de sentiment.
L'évolution des coefficients saisonniers de l'indicateur de sentiment calculé à partir modèle de référence suggère que la saisonnalité de cette série est faible.
L'étude de ceux de l'indicateur calculé à partir du modèle \emph{word2vec} suggère que les tweets sont globalement plus pessimistes en février et mars, et plus optimistes en avril, août et décembre.

\begin{figure}[htp]
{\centering\includegraphics[width = \textwidth]{img/rmd-graphCoefCVS-1}}
\captionsetup{margin=0cm,format=hang,justification=justified}
\caption{Commposantes saisonnières des indicateurs mensuels de sentiment construits à partir de notre modèle \emph{word2vec} et à partir de notre modèle de référence.}\label{fig:coefcvscjo}
\footnotesize

\emph{La composante saisonnière a été estimée par la méthode X-12ARIMA.}
\end{figure}

\newpage

\small

\nocite{*}

\begin{thebibliography}{999}
\bibitem[Bengio \emph{et al} (2003)]{Bengio} Bengio, Y., Ducharme, R., Vincent, P., Janvin, C. (2003). A Neural Probabilistic Language Model. JMLR, 3:1137–1155. \url{https://papers.nips.cc/paper/1839-a-neural-probabilistic-language-model.pdf}.
\bibitem[Bojanowski et al. (2017)]{Bojanowski} Bojanowski, P., Grave, E., Joulin, A., \& Mikolov, T. (2017). Enriching word vectors with subword information. Transactions of the Association for Computational Linguistics, 5, 135-146. \url{https://www.mitpressjournals.org/doi/abs/10.1162/tacl_a_00051}.
\bibitem[Boumedyen Billami \& Gala (2017)]{Boumedyen} Boumedyen Billami, M.,  Gala, N (2017). Création et validation de signatures sémantiques : application à la mesure de similarité sémantique et à la substitution lexicale. TALN 2017. \url{https://hal.archives-ouvertes.fr/hal-01528117/document}.
\bibitem[Candela et al. (2009)]{Candela} Candela, J. Q., Sugiyama, M., Schwaighofer, A., \& Lawrence, N. D. (2009) Dataset shift in machine learning. The MIT Press, 1, 5. \url{http://www.acad.bg/ebook/ml/The.MIT.Press.Dataset.Shift.in.Machine.Learning.Feb.2009.eBook-DDU.pdf}.
\bibitem[Hutter, Hoos \& Leyton-Brown (2014)]{Hutter} Hutter, F., Hoos, H., Leyton-Brown, K., (2014). An Efficient Approach for Assessing Hyperparameter Importance. PMLR 32(1):754-762. \url{http://proceedings.mlr.press/v32/hutter14.pdf}.
\bibitem[Jurafsky \& Martin (2019)]{Jurafsky} Jurafsky, D., Martin, J. H. (2019). Speech and Language Processing (3rd ed. draft). Prentice Hall. \url{https://web.stanford.edu/~jurafsky/slp3/edbook_oct162019.pdf}.
\bibitem[Ladiray (2018)]{L2018} Ladiray, D. (2018), Calendar effects, in Handbook on Seasonal Adjustment, edited by G. L. Mazzi, co-edited by D. Ladiray, European Union, Luxembourg. \url{ec.europa.eu/eurostat/web/products-manuals-and-guidelines/-/KS-GQ-18-001}.
\bibitem[Levy \& Golberg (2015)]{Levy} Levy, O., Golberg, Y. (2015). Neural Word Embedding as Implicit Matrix Factorization.
\url{https://papers.nips.cc/paper/5477-neural-word-embedding-as-implicit-matrix-factorization.pdf}.
\bibitem[Levy \& Golberg (2014)]{Levy2} Levy, O., Golberg, Y. (2014). Dependency-based word embeddings. ACL. \url{http://papers.nips.cc/paper/5477-neural-word-embedding-as-implicit-matrix-factorization.pdf}.
\bibitem[Lytras \emph{et al} (2007)]{lytras} Lytras, D. P., Feldpausch, R. M., Bell, W. R (2007). Determining Seasonality: A Comparison of Diagnostics from X-12-ARIMA, U.S. Census Bureau, presented at ICES III, June, 2007. \url{https://www.census.gov/ts/papers/ices2007dpl.pdf}.
\bibitem[Mikolov \emph{et al} (2013a)]{Mikolov} Mikolov, T.,  Chen, K., Corrado, G., Dean, J. (2013a). Efficient Estimation of Word Representations in Vector Space. arXiv:1301.3781. \url{https://arxiv.org/pdf/1301.3781.pdf}.
\bibitem[Mikolov \emph{et al} (2013b)]{MikolovNS} Mikolov, T., Sutskever, I., Chen, K., Corrado, G. S., and Dean, J. (2013b). Distributed representations of words and phrases and their compositionality, arXiv:1310.4546. \url{https://arxiv.org/pdf/1310.4546.pdf}.
\bibitem[Pennington, Socher \& Manning (2014)]{Pennington} Pennington, J., Socher, R., Manning, C. D., (2014).  Glove: global vectors for word representation. Proc. of EMNLP,1532 – 1543. \url{https://www.aclweb.org/anthology/D14-1162.pdf}.
\bibitem[{\v R}eh{\r u}{\v r}ek \& Sojka (2010)]{Rehurek} {\v R}eh{\r u}{\v r}ek, R.,  Sojka, P. (2010). Software Framework for Topic Modelling with Large Corpora. Proceedings of LREC 2010 workshop New Challenges for NLP Frameworks. p. 46--50, 5 pp. ISBN 2-9517408-6-7. \url{https://is.muni.cz/publication/884893/en}.
\bibitem[Rubenstein \& Goodenough (1965)]{Rubenstein} Rubenstein, H.,  Goodenough, J. B. (1965). Contextual Correlates of Synonymy. Commun. ACM, 8 (10), 627–633. \url{https://dl.acm.org/doi/10.1145/365628.365657}.
\bibitem[Schakel \& Wilson (2015)]{Schakel} Schakel, A. M., Wilson, B. J. (2015). Measuring Word Significance using Distributed Representations of Words. arXiv:1508.02297. \url{https://arxiv.org/pdf/1508.02297v1.pdf}.
\bibitem[Tormene \emph{et al} (2008)]{dtw} Tormene P, Giorgino T, Quaglini S, Stefanelli M (2008). “Matching
Incomplete Time Series with Dynamic Time Warping: An Algorithm and an Application to Post-Stroke Rehabilitation.” \emph{Artificial Intelligence
in Medicine}, \textbf{45}(1), 11-34. doi: 10.1016/j.artmed.2008.11.007, \url{https://doi.org/10.1016/j.artmed.2008.11.007}.
\end{thebibliography}

\end{document}
